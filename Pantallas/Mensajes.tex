\subsubsection{Catálogo de mensajes}
\label{sec:mensajes}

\paragraph{\textcolor[rgb]{0, 0, 0.545098}{MSG01 Operación realizada exitosamente}} \hspace{1cm} \\
\label{msj:MSG01}
\rule[3mm]{16.59cm}{0.1mm} \vspace{1mm}
\textbf{Tipo:} Notificación.\\
\textbf{Objetivo:} Notificar al usuario que la acción solicitada fue realizada exitosamente.\\
\textbf{Redacción:} ENTIDAD ha sido OPERACIÓN exitosamente.\\
\textbf{Parámetros:} El mensaje se muestra con base en los siguientes parámetros:
\begin{itemize} \itemsep1pt \parskip0pt \parsep0pt
	\item ENTIDAD: Es una entidad sobre la cual se ha realizado  la acción.
	\item OPERACIÓN: Es la acción que el actor solicita realizar.
\end{itemize}
\textbf{Ejemplo:} \textit{El Practicante} ha sido \textit{registrado} exitosamente.\\

\paragraph{\textcolor[rgb]{0, 0, 0.545098}{MSG02 No existen movimientos para registrar una rutina}} \hspace{1cm} \\
\label{msj:MSG02}
\rule[3mm]{16.59cm}{0.1mm} \vspace{1mm}
\textbf{Tipo:} Notificación.\\
\textbf{Objetivo:} Notificar al usuario que aún no existe información registrada en la herramienta.\\
\textbf{Redacción:} No existen registros en los catálogos de ENTIDAD en la herramienta.\\
\textbf{Parámetros:} El mensaje se muestra con base en los siguientes parámetros:
\begin{itemize} \itemsep1pt \parskip0pt \parsep0pt
	\item ENTIDAD: Específica la entidad sobre la cual se está realizando la consulta.
\end{itemize}
\textbf{Ejemplo:} No existen \textit{movimientos} capturados en la herramienta.\\

\paragraph{\textcolor[rgb]{0, 0, 0.545098}{MSG03 Envío de contraseña realizado exitosamente}} \hspace{1cm} \\
\label{msj:MSG03}
\rule[3mm]{16.59cm}{0.1mm} \vspace{1mm}
\textbf{Tipo:} Notificación.\\
\textbf{Objetivo:} Notificar al usuario que la contraseña registrada ha sido enviada al correo ingresado.\\
\textbf{Redacción:} La contraseña ha sido enviada al correo ingresado.\\

\paragraph{\textcolor[rgb]{0, 0, 0.545098}{MSG04 Confirmar la eliminación de un elemento}} \hspace{1cm} \\
\label{msj:MSG04}
\rule[3mm]{16.59cm}{0.1mm} \vspace{1mm}
\textbf{Tipo:} Confirmación.\\
\textbf{Objetivo:} Informar al usuario que está a punto de eliminar un elemento y que se necesita su aprobación para ello.\\
\textbf{Redacción:} Se eliminará ENTIDAD seleccionado(a). ¿Está seguro que desea continuar?\\
\textbf{Parámetros:} El mensaje se muestra con base en los siguientes parámetros:
\begin{itemize} \itemsep1pt \parskip0pt \parsep0pt
	\item ENTIDAD: Es una entidad sobre la cual se realizará la acción.
\end{itemize}
\textbf{Ejemplo:} Se eliminará \textit{la rutina} seleccionada. ¿Está seguro que desea continuar?.\\

\paragraph{\textcolor[rgb]{0, 0, 0.545098}{MSG05 Confirmar la eliminación de un Practicante}} \hspace{1cm} \\
\label{msj:MSG05}
\rule[3mm]{16.59cm}{0.1mm} \vspace{1mm}
\textbf{Tipo:} Confirmación.\\
\textbf{Objetivo:} Informar al usuario que está a punto de eliminar un Practicante y todos los datos asociados a él y que se necesita su aprobación para ello.\\
\textbf{Redacción:} Se eliminará el Practicante NOMBRE, todas sus rutinas asignadas y TODA la información asignada a él. ¿Está seguro que desea continuar?\\
\textbf{Parámetros:} El mensaje se muestra con base en los siguientes parámetros:
\begin{itemize} \itemsep1pt \parskip0pt \parsep0pt
	\item NOMBRE: Es un nombre del Practicante.
\end{itemize}
\textbf{Ejemplo:} Se eliminará el Practicante \textit{Daniel Sánchez}, todas sus rutinas asignadas y TODA la información asignada a él.\\

\paragraph{\textcolor[rgb]{0, 0, 0.545098}{MSG06 Confirmación de envío de desempeño}} \hspace{1cm} \\
\label{msj:MSG06}
\rule[3mm]{16.59cm}{0.1mm} \vspace{1mm}
\textbf{Tipo:} Confirmación.\\
\textbf{Objetivo:} Informar al usuario que está a punto de enviar el desempeño de la rutina realizada y que necesita su aprobación para ello.\\
\textbf{Redacción:} ¿Está seguro que desea enviar el desempeño? Una vez que envíe ya no podrá repetir la rutina.\\

\paragraph{\textcolor[rgb]{0, 0, 0.545098}{MSG07 Confirmación de asignación de rutinas}} \hspace{1cm} \\
\label{msj:MSG07}
\rule[3mm]{16.59cm}{0.1mm} \vspace{1mm}
\textbf{Tipo:} Confirmación.\\
\textbf{Objetivo:} Informar al usuario que está a punto de asignar rutinas y no podrá modificar esta asignación en la semana correspondiente.\\
\textbf{Redacción:} Sólo se puede asignar rutinas una vez por semana, ¿Está seguro que desea continuar?\\

\paragraph{\textcolor[rgb]{0, 0, 0.545098}{MSG08 Envío de contraseña}} \hspace{1cm} \\
\label{msj:MSG08}
\rule[3mm]{16.59cm}{0.1mm} \vspace{1mm}
\textbf{Tipo:} Correo.\\
\textbf{Objetivo:} Informar al usuario cual es su contraseña para iniciar sesión en la herramienta.\\
\textbf{Asunto:} Envío de contraseña de la herramienta de apoyo al entrenamiento de Karate Do\\
\textbf{Redacción:} \\
	Bienvenido(a) a la herramienta de apoyo al entrenamiento de Karate Do.\\

	Estimado(a) NOMBRE.\\

	Se le envían sus datos de inicio de sesión a la herramienta.\\

	Nombre de usuario: USUARIO
	
	Contraseña: CONTRASEÑA\\
	
	Para mayor seguridad favor de cambiar su contraseña lo antes posible.\\
\textbf{Parámetros:} El mensaje se muestra con base en los siguientes parámetros:
\begin{itemize} \itemsep1pt \parskip0pt \parsep0pt
	\item NOMBRE: Es el nombre completo del Practicante.
	\item USUARIO: Es el nombre de usuario para ingresar a la herramienta.
	\item CONTRASEÑA: Es la contraseña generada por la herramienta.
\end{itemize}

\paragraph{\textcolor[rgb]{0, 0, 0.545098}{MSG09 No se puede eliminar el elemento}} \hspace{1cm} \\
\label{msj:MSG09}
\rule[3mm]{16.59cm}{0.1mm} \vspace{1mm}
\textbf{Tipo:} Notificación.\\
\textbf{Objetivo:} Informar al usuario que el elemento seleccionado no se puede eliminar porque esta asignado o forma parte de otro elemento.\\
\textbf{Redacción:} No se puede eliminar ENTIDAD porque está asignado(a) a ELEMENTO.\\
\textbf{Parámetros:} El mensaje se muestra con base en los siguientes parámetros:
\begin{itemize} \itemsep1pt \parskip0pt \parsep0pt
	\item ENTIDAD: Es una entidad sobre la cual se ha realizado la acción.
	\item ELEMENTO: Es un elemento conformado a partir de diferentes entidades.
\end{itemize}
\textbf{Ejemplo:} No se puede eliminar \textit{la rutina} porque está asignada a \textit{un Practicante}.\\

\paragraph{\textcolor[rgb]{0, 0, 0.545098}{MSG10 Usuario y/o contraseña incorrectos}} \hspace{1cm} \\
\label{msj:MSG10}
\rule[3mm]{16.59cm}{0.1mm} \vspace{1mm}
\textbf{Tipo:} Error.\\
\textbf{Ubicación:} Campo. Se muestra a la derecha del campo.\\
\textbf{Objetivo:} Informar al usuario que el nombre de usuario y/o contraseña son incorrectos.\\
\textbf{Redacción:} El nombre de usuario y/o contraseña son incorrectos. Favor de verificarlo.\\

\paragraph{\textcolor[rgb]{0, 0, 0.545098}{MSG11 Registro repetido}} \hspace{1cm} \\
\label{msj:MSG11}
\rule[3mm]{16.59cm}{0.1mm} \vspace{1mm}
\textbf{Tipo:} Error.\\
\textbf{Objetivo:} Informar al usuario que ya existe un registro con los mismos datos.\\
\textbf{Redacción:} Error, ya existe ENTIDAD con el mismo nombre en la herramienta.\\
\textbf{Parámetros:} El mensaje se muestra con base en los siguientes parámetros:
\begin{itemize} \itemsep1pt \parskip0pt \parsep0pt
	\item ENTIDAD: Es una entidad sobre la cual se ha realizado  la acción.
\end{itemize}
\textbf{Ejemplo:} Error, ya existe \textit{un Practicante} con el mismo nombre en la herramienta.\\

\paragraph{\textcolor[rgb]{0, 0, 0.545098}{MSG12 Falta un dato obligatorio para efectuar la operación solicitada}} \hspace{1cm} \\
\label{msj:MSG12}
\rule[3mm]{16.59cm}{0.1mm} \vspace{1mm}
\textbf{Tipo:} Error.\\
\textbf{Ubicación:} Campo. Se muestra a la derecha del campo.\\
\textbf{Objetivo:} Informar al usuario la omisión de algún dato obligatorio para realizar la operación solicitada.\\
\textbf{Redacción:} Campo obligatorio.\\

\paragraph{\textcolor[rgb]{0, 0, 0.545098}{MSG13 Formato incorrecto}} \hspace{1cm} \\
\label{msj:MSG13}
\rule[3mm]{16.59cm}{0.1mm} \vspace{1mm}
\textbf{Tipo:} Error.\\
\textbf{Ubicación:} Campo. Se muestra a la derecha del campo.\\
\textbf{Objetivo:} Informar al usuario que el dato ingresado en alguno de los campos de registro no cumple con el formato correcto.\\
\textbf{Redacción:} Formato incorrecto.\\

\paragraph{\textcolor[rgb]{0, 0, 0.545098}{MSG14 Las contraseñas no coinciden}} \hspace{1cm} \\
\label{msj:MSG14}
\rule[3mm]{16.59cm}{0.1mm} \vspace{1mm}
\textbf{Tipo:} Error.\\
\textbf{Objetivo:} Informar al usuario cuando desea modificar su contraseña que tanto la nueva contraseña como su confirmación no coinciden.\\
\textbf{Redacción:} Las contraseñas ingresadas deben coincidir.\\

\paragraph{\textcolor[rgb]{0, 0, 0.545098}{MSG15 Contraseña vacía}} \hspace{1cm} \\
\label{msj:MSG15}
\rule[3mm]{16.59cm}{0.1mm} \vspace{1mm}
\textbf{Tipo:} Error.\\
\textbf{Objetivo:} Informar al usuario cuando desea modificar su contraseña que tanto la nueva contraseña no puede estar vacía.\\
\textbf{Redacción:} La contraseña no puede estar vacía.\\

\paragraph{\textcolor[rgb]{0, 0, 0.545098}{MSG16 Número de elementos válidos para el registro de una rutina}} \hspace{1cm} \\
\label{msj:MSG16}
\rule[3mm]{16.59cm}{0.1mm} \vspace{1mm}
\textbf{Tipo:} Error.\\
\textbf{Objetivo:} Informar al usuario que debe asignar entre un mínimo y un máximo de ejercicios y movimientos en la rutina a registrar.\\
\textbf{Redacción:} Debes seleccionar entre MÍNIMO y MÁXIMO ENTIDAD.\\
\textbf{Parámetros:} El mensaje se muestra con base en los siguientes parámetros:
\begin{itemize} \itemsep1pt \parskip0pt \parsep0pt
	\item MÍNIMO. Es el número mínimo de repeticiones por cada entidad.
	\item MÁXIMO. Es el número máximo de repeticiones por cada entidad.
	\item ENTIDAD. Es un ejercicio de calentamiento o movimiento de técnica.
\end{itemize}
\textbf{Ejemplo:} Debes seleccionar entre \textit{3} y \textit{5} ejercicios de calentamiento.\\

\paragraph{\textcolor[rgb]{0, 0, 0.545098}{MSG17 Día de asignación incorrecto}} \hspace{1cm} \\
\label{msj:MSG17}
\rule[3mm]{16.59cm}{0.1mm} \vspace{1mm}
\textbf{Tipo:} Error.\\
\textbf{Objetivo:} Informar al usuario que el día de asignación de rutinas debe ser lunes.\\
\textbf{Redacción:} Sólo puedes asignar rutinas los días lunes.\\

\paragraph{\textcolor[rgb]{0, 0, 0.545098}{MSG18 Número mínimo de rutinas asignadas}} \hspace{1cm} \\
\label{msj:MSG18}
\rule[3mm]{16.59cm}{0.1mm} \vspace{1mm}
\textbf{Tipo:} Error.\\
\textbf{Objetivo:} Informar al usuario que debe ingresar un mínimo de rutinas para asignarlas a un Practicante.\\
\textbf{Redacción:} Debe asignar por lo menos una rutina.\\

\paragraph{\textcolor[rgb]{0, 0, 0.545098}{MSG19 Rutinas diferentes por día}} \hspace{1cm} \\
\label{msj:MSG19}
\rule[3mm]{16.59cm}{0.1mm} \vspace{1mm}
\textbf{Tipo:} Error.\\
\textbf{Objetivo:} Informar al usuario que no puede realizar más de una rutina diferente por día.\\
\textbf{Redacción:} No puedes realizar más de una rutina diferente el mismo día.\\

\paragraph{\textcolor[rgb]{0, 0, 0.545098}{MSG20 Número máximo de rutinas registradas permitidas}} \hspace{1cm} \\
\label{msj:MSG20}
\rule[3mm]{16.59cm}{0.1mm} \vspace{1mm}
\textbf{Tipo:} Error.\\
\textbf{Objetivo:} Informar al usuario que ya ha alcanzado el número de registro de rutinas permitidas en la herramienta.\\
\textbf{Redacción:} Has alcanzado el número máximo de registro de rutinas permitidas.\\

\paragraph{\textcolor[rgb]{0, 0, 0.545098}{MSG21 No se puede enviar el desempeño de un rutina no realizada}} \hspace{1cm} \\
\label{msj:MSG21}
\rule[3mm]{16.59cm}{0.1mm} \vspace{1mm}
\textbf{Tipo:} Error.\\
\textbf{Objetivo:} Informar al usuario que no se puede enviar el desempeño de una rutina que no se ha realizado previamente.\\
\textbf{Redacción:} No puedes enviar el desempeño de una rutina que no has realizado.\\

\paragraph{\textcolor[rgb]{0, 0, 0.545098}{MSG22 Número máximo de practicantes registrados permitidos}} \hspace{1cm} \\
\label{msj:MSG22}
\rule[3mm]{16.59cm}{0.1mm} \vspace{1mm}
\textbf{Tipo:} Error.\\
\textbf{Objetivo:} Informar al usuario que ya ha alcanzado el número de registro de practicantes permitidos en la herramienta.\\
\textbf{Redacción:} Has alcanzado el número máximo de registro de practicantes permitidos.\\

\paragraph{\textcolor[rgb]{0, 0, 0.545098}{MSG23 Asignación de rutinas por día}} \hspace{1cm} \\
\label{msj:MSG23}
\rule[3mm]{16.59cm}{0.1mm} \vspace{1mm}
\textbf{Tipo:} Error.\\
\textbf{Objetivo:}  Informar al usuario que no puede asignar más rutinas a ese Practicante hasta la siguiente semana.\\
\textbf{Redacción:} No puedes asignar más rutinas a un Practicante en un mismo día.\\

\paragraph{\textcolor[rgb]{0, 0, 0.545098}{MSG24 Elementos mostrados incorrectamente}} \hspace{1cm} \\
\label{msj:MSG24}
\rule[3mm]{16.59cm}{0.1mm} \vspace{1mm}
\textbf{Tipo:} Error.\\
\textbf{Objetivo:} Informar al usuario que debido a un problema de conexión no se han mostrado los elementos de la interfaz de forma correcta.\\
\textbf{Redacción:} Error de conexión. No se pudieron cargar los elementos adecuadamente.\\

\paragraph{\textcolor[rgb]{0, 0, 0.545098}{MSG25 Error de conexión}} \hspace{1cm} \\
\label{msj:MSG25}
\rule[3mm]{16.59cm}{0.1mm} \vspace{1mm}
\textbf{Tipo:} Error.\\
\textbf{Objetivo:} Informar al usuario que no puede realizar la conexión con la herramienta.\\
\textbf{Redacción:} No se puede realizar la conexión con la herramienta. Favor de verificar tu conexión a internet.\\

% \paragraph{\textcolor[rgb]{0, 0, 0.545098}{}} \hspace{1cm} \\
% \label{msj:}
% \rule[3mm]{16.59cm}{0.1mm} \vspace{1mm}
% \textbf{Tipo:} .\\
% \textbf{Objetivo:} \\
% \textbf{Redacción:} \\
% \textbf{Parámetros:} El mensaje se muestra con base en los siguientes parámetros:
% \begin{itemize} \itemsep1pt \parskip0pt \parsep0pt
	% \item 
	% \item 
% \end{itemize}
% \textbf{Ejemplo:} \textit{} \textit{} .\\

\clearpage