\subsection{Gestión de rutinas}

La gestión de rutinas consiste en altas, bajas y modificaciones. El propósito de incluir una gestión de rutinas es para brindar la posibilidad de crear rutinas de entrenamiento específicas, y que se adecuen a las necesidades de cada Practicante, introduciendo una valiosa herramienta para el entrenamiento personalizado pues en las clases grupales esto no siempre es posible.\\

Las altas de nuevas rutinas se pueden realizar únicamente cuando se ha capturado el mínimo de los movimientos del catálogo de acuerdo a lo específicado en la regla de negocio \nameref{rn:RNR15}. Las rutinas se componen de ejercicios de calentamiento y movimientos de técnica, así como de sus identificadores. Los movimientos que conforman una rutina son elegidos de acuerdo a los criterios del experto y los ejercicios de calentamiento deben ser elegidos conforme a los movimientos elegidos con el fin de evitar lesiones cuando los Practicantes realicen las rutinas. Se propone realizar como máximo 24 registros contemplando que son necesarias 24 clases impartidas a lo largo de 2 meses para el cambio de cinta, de blanca a blanca avanzada.\\

Las modificaciones de una rutina consisten en modificar los identificadores de la rutina seleccionada, los cuales consisten en una imagen alusiva, y el nombre de la rutina, dichas modificaciones se podrán realizar siempre y cuando no exista una rutina registrada previamente con el mismo nombre.\\

Debido a la sencillez para crear rutinas de entrenamiento, sólo se permite la eliminación de una rutina si ésta no se encuentra asignada a un Practicante.\\

\subsubsection{Asignación de rutinas}

Una de las problemáticas descritas en este trabajo terminal, señala que la falta de tiempo para asistir a actividades previamente programadas y tener que desplazarse a un lugar en específico, son las principales causas para que las personas desistan de realizar una actividad física deportiva. Como una posible solución a esta problemática se introdujo en este trabajo el término de entrenamiento a distancia, en el cual se puede realizar el entrenamiento de una disciplina deportiva como el Karate Do sin tener que asistir a una clase específica, de manera que se puede hacer uso del tiempo libre de los Practicantes sin que los mismos tengan que someterse a presiones que involucren tiempo y distancia, así como un elevado costo. \\

El Entrenador puede asignar las rutinas creadas previamente a sus estudiantes dentro de los límites de una fecha de asignación establecida. Se establece una fecha de asignación en la cual es posible asignar hasta tres rutinas semanalmente pues un entrenamiento promedio consiste en tres sesiones de entrenamiento semanales. Fuera de esta fecha de asignación no se pueden realizar más asignaciones para evitar la sobrecarga de trabajo físico para un Practicante.\\

\subsubsection{Catálogos de ejercicios de calentamiento y movimientos de técnica}

Debido a que la realización de ejercicios de calentamiento previos a la práctica de alguna actividad física es de vital importancia para evitar lesiones, se establece un catálogo de ejercicios de calentamiento propuestos por la CONADE en sus programas de activación física (ver sección \ref{sec:Calentamiento} \nameref{sec:Calentamiento}, el cual está compuesto de al menos un ejercicio por cada parte del cuerpo que se involucra en la realización de los movimiento de técnica. Para la correcta realización del calentamiento previo, el mínimo de repeticiones de cada ejercicio se establece en 8 y el máximo en 20 repeticiones, las cuales serán elegidas por el Entrenador.\\

El catálogo de movimientos de técnica se compone de los movimientos de técnica de cinta blanca de Karate Do. Este catálogo se encuentra inicialmente vacío y se actualiza una vez que el Entrenador realiza la captura de los movimientos del catálogo. Los movimientos se dividen por hemisferio, es decir, en aquellos movimientos que se puedan realizar utilizando ambos hemisferios (acción en espejo), en realidad constituyen dos movimientos en el catálogo (uno por cada hemisferio). Ver sección \ref{sec:Karate-Do} \nameref{sec:Karate-Do} para más detalles.\\