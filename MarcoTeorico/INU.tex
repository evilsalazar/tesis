%------------------------------------------------------------------------------------
\section{Interfaz Natural de Usuario (INU)}

Existen diferentes definiciones de Interfaz natural de usuario tal como lo puntualizan los siguientes autores: \\

Una Interfaz natural de usuario es una interfaz de usuario diseñada para reutilizar habilidades existentes para interactuar directamente con contenido. ``Interfaz natural de usuario es la nueva forma de pensamiento acerca de como podemos interactuar con dispositivos de computación" \cite{Blake}. \\

La Interfaz natural de usuario (INU) es una evolución de la interfaz gráfica de usuario (GUI) y surge como un mecanismo de interacción hombre-máquina que permite establecer una  comunicación con sistemas computacionales a través de periféricos que pueden recibir instrucciones e información. La INU tiene algunas consideraciones adicionales las cuales se enfocan en hacer uso de una comunicación de manera natural con el ser humano, haciendo uso del captar información en tiempo real logrando una interacción corporal de manera directa, sin utilizar un periférico que actúe como intermediario para la entrada de información \cite{Gomez}. \\

La INU fundamenta sus principios en lo siguiente: debe ser directo, intuitivo e invisible al usuario (o hacerse invisible por medio de interacciones sencillas explicadas al usuario). No tiene que ser aprendido, porque está basado en elementos naturales, es decir, en los comandos utilizados en los comportamientos humanos habituales. Integra armoniosamente un balance entre fisiología y kinesiología, creando una interfaz para dedos vivos, no para cursores \cite{Sarajevo}. \\

\textbf{¿Qué significa natural?} \\

Una interfaz natural es aquella que ``explota las habilidades que hemos adquirido a través del tiempo vivido en el mundo" \cite{Buxton}. \\

La descripción anterior es interesante por dos razones. Primero, relaciona el concepto natural con la idea de rehusar habilidades existentes. Segundo, hace explícito que estas habilidades no son solamente habilidades innatas con las cuales nacimos. Natural significa usar habilidades innatas mas habilidades aprendidas y que hemos desarrollado a través de la interacción con nuestro propio ambiente natural cada día de la vida. \\

\textbf{¿Qué es una habilidad simple?} \\

Las habilidades simples solo dependen de las habilidades innatas aprendidas. Esto limita su complejidad, lo que también significa que son fáciles de aprender, tienen una baja carga cognitiva, y pueden ser rehusadas y adaptadas para muchos objetivo sin mucho esfuerzo. El proceso de aprendizaje de habilidades simples es típicamente muy veloz y requiere poca o ninguna práctica para alcanzar un adecuado nivel de competencia. En muchas ocasiones, este aprendizaje se puede alcanzar simplemente observando a alguien mas demonstrar la habilidad una o dos veces \cite{Blake}.

\subsection{Características de una Interfaz natural de usuario} 

\textbf{Centrada en el usuario} \\

Se toman como partida las necesidades cambiantes de la interfaz de usuario, de manera que se modifica la interfaz de usuario de la forma externa y los mecanismos internos para satisfacer las necesidades de los diferentes usuarios, lo cual es llamado diseño centrado en el usuario. La tecnología de reconocimiento de voz no específica de discursos humanos permitirá a las computadoras comprender las demandas de las personas, esta es una interfaz de entrada importante. \\

La tecnología fisheye observa la posición de los alrededores de un contenido con respecto a una posición en la pantalla, la cual es aumentada y es llamada observación amigable del usuario. En los sistemas tradicionales humano-maquina, las personas son consideradas como los operadores, es decir, personas que se adaptan a la máquina; en los sistemas generales hombre-máquina, las personas son conocidas como usuarios, pueden dialogar con la máquina, pero no tienen el control activo; en los sistemas de realidad virtual, las personas son participantes activos, donde la máquina responderá a diferentes acciones humanas. \\

\textbf{Multicanal} \\

Las interfaces multicanal tienen la intención de hacer un uso completo de uno o más de los canales sensores y motores, para capturar las características complementarias de los propósitos de los usuarios, para mejorar la naturalidad de la interacción hombre-máquina. \\

Los sentidos sensoriales humanos son la visión, el oído, el tacto, el olfato y el gusto; los canales de gestos humanos son las manos, la boca, los ojos, la cabeza, los pies, el cuerpo y así sucesivamente. Ahora, para las operaciones computacionales, los ojos humanos y las manos están muy cansados y la eficiencia no es alta \cite{Weiyuan}. Si escuchamos a, por ejemplo, la coordinación mano-ojo y otras acciones multicanal, se puede lograr una interacción natural y una comunicación humano-máquina eficiente así como brindar la posibilidad de elegir el mejor canal de respuesta posible, de tal manera que no se haga una sobrecarga sobre dichos canales. \\

\textbf{Inexacta} \\

La tecnología interactiva precisa, es una tecnología que puede ser usada para explicar completamente el propósito de las interacciones del usuario. El teclado y el ratón son necesarios para ingresar datos con precisión, pero las acciones de las personas o pensamientos no son muy precisos. Las computadores deben entender las peticiones de las personas e incluso corregir sus errores. Una interfaz que sea inteligente es una orientación importante. \\

\textbf{Alto ancho de banda} \\

Ahora las salidas de una computadora tienen que ser rápidas, con un continuo despliegue de imágenes a color y de una gran cantidad de información. Pero las personas aún utilizan como entrada el teclado en una que otra ocasión, por lo cual el ancho de banda es muy bajo. Las Interfaces naturales de usuario deben soportar un ancho de banda de entrada alto y una rápida importación de grandes cantidades de información. La entrada y entendimiento de voz, imagen, y la postura es una orientación de desarrollo para la actualidad y el futuro. \\

\textbf{Interacciones basadas en voz} \\

El lenguaje ha sido reconocido por mucho tiempo como el flujo más natural, conveniente y eficiente para el intercambio de información. En la vida diaria la comunicación humana es en un 75\% ejecutada por voz \cite{Weiyuan}. Los resultados muestran que existen muchas ventajas sobre los canales auditivos, como que la detección de canales auditivos es más rápida que la velocidad de detección de señales visuales. El cambio de un sonido personal es extremadamente sensible; la información auditiva y visual puede proporcionar el acceso a más personas hacia la existencia de un fuerte sentido de realismo, etc. Por lo tanto, el canal auditivo es el mas importante canal de interacción entre una computadora y otro dispositivo de información \cite{Weiyuan}. \\

La interacción por voz es una interacción con tecnología computacional para estudiar el como las personas interactúan a través de voz o voz sintetizadas por máquina. Esto involucra diferentes disciplinas como lingüística, psicología, ergonomía y tecnologías de la computación; al mismo tiempo es una visión hacía el futuro dirigido a la interacción por voz, su diseño y desarrollo. \\

Los sistemas interactivos por voz típicamente toman dos aproximaciones: una es basada en una tecnología de reconocimiento y entendimiento, que depende principalmente del sistema de audio con el que se interactúa; el otro es el uso de una tecnología de voz y de sistemas combinados en otras formas para interactuar con el sistema. De esta forma, la voz ya no es dominante sino sólo una parte de un sistema interactivo. \\

\textbf{Interacción basada en imagen} \\

La interacción por imagen simplemente es una computadora basada en comportamiento humano que entiende una imagen y entonces reacciona. \\

En el presente, los sistemas de visión artificial se pueden dividir en 3 niveles: el procesamiento de imagen (al más bajo nivel), el reconocimiento de imágenes (a un alto nivel) y la percepción de imágenes (al más alto nivel). El procesamiento de imágenes es el proceso de ingresar una imagen como entrada y de tener una imagen a la salida \cite{Weiyuan}. El reconocimiento de patrones, se interesa principalmente en la detección de objetivos en la imagen y su medición, para obtener información objetiva, con el fin de establecer la descripción de la imagen. Esencialmente es el proceso de llevar una imagen a datos. \\

La percepción de imagen se centra en estudiar aún más la naturaleza de la imagen de destino y sus relaciones mutuas sobre la base del reconocimiento de imágenes, y viene para entender el significado del contenido de una imagen e interpretación de la escena original. En la percepción de imágenes, la entrada es una imagen y la salida es una interpretación de dicha imagen. \\

\textbf{Interacciones basadas en comportamiento} \\

En el proceso de intercambio, en adición a la interacción por voz, las personas frecuentemente utilizan el lenguaje corporal, el cual involucra movimientos del cuerpo para expresar su actitud y significado, por lo tanto, es un proceso basado en interacciones humanas. El método de interacciones por acciones humanas puede no solamente mejorar las habilidades de lenguajes, sino también, puede jugar el rol de interacción que la voz no puede. \\

El comportamiento de la interacción humano-computadora es el reconocimiento de comportamiento humano a través de posicionamiento, seguimiento, movimiento y expresión de las partes del cuerpo para entender las acciones y el comportamiento, y responder con un proceso inteligente de retroalimentación. \\

Las interacciones basadas en comportamiento brindan una nueva forma de interacción. El comportamiento del usuario puede ser predecido por la computadora y así conocer las necesidades de los usuarios. Por ejemplo: con un seguimiento de la atención de las personas, se puede determinar la intención de los usuarios para visitar un sitio web especifico o la necesidad de una llamada; cuando el usuario entra en una habitación, la computadora responde enviándole un correo electrónico, si el usuario niega con la cabeza, el equipo considera que el usuario no desea leer el mensaje.