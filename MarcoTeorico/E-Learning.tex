%------------------------------------------------------------------------------------
\section{E-Learning}

Con la población en Internet, la demanda de e-Learning ha incrementado recientemente \cite{Chang}. E-Learning representa una aproximación casi ideal para el desarrollo flexible y rentable de competencias desde que pueden ser usadas sin restricciones relacionadas con las ubicaciones físicas y el tiempo de uso \cite{Cukusic}. Una de las principales características del e-Learning es la capacidad de integrar diferentes medios, como texto, imágenes, audio, animaciones y video para crear instrucciones multimedia, promoviendo los intereses de lectura y disposición del aprendiz \cite{Vichuda}. De manera que, la investigación ha mostrado que el diseño de recursos multimedia es costoso y no tienen efectos consistentes en promover el desarrollo del aprendizaje. Por ejemplo, el material didáctico multimedia es menos importante que el proceso y la función del sistema, con el fin de producir un efecto significativo en la comprensión del contenido educativo, aunque de esta forma llame más la atención del aprendiz \cite{Bartscha}. Además, algunas investigaciones muestran que demasiados elementos multimedia innecesarios pueden distraer a los aprendices \cite{Clark}. También, el diseño y desarrollo de recursos multimedia didacticas es costoso, aunque el rapido progreso de las tecnologías de la computación e Internet hacen tal esfuerzo posible \cite{Sun}. No obstante los elementos multimedia son muy importantes para el usuario o aprendiz para utilizar una sistema basado en e-Learning. Por lo tanto, la forma de desarrollar y apoyar los contenidos multimedia efectivos de aprendizaje basadas en el mejor ajuste del alumno de acuerdo con el curso de aprendizaje se está convirtiendo en un problema importante de e-Learning. Con respecto a la evaluación de la evidencia científica, que distribuyen e identifican la calidad del sistema de enseñanza. \\

La educación tradicional en salones de clase no siempre satisface todas las necesidades del nuevo mundo para que el aprendizaje sea permanente. El e-Learning, hace referencia al aprendizaje vía Internet, provee a las personas una forma flexible y personalizada de aprender. Ofrece oportunidades de aprendizaje en demanda y reduce los costos \cite{Zhang}. \\

Enseñar y aprender ya no son exclusivos dentro de un salón de clases tradicional \cite{NC}. Los métodos de aprendizaje necesitan ser más portables y flexibles. El nuevo enfoque del aprendizaje a distancia es construir una infraestructura de costo efectivo que permita el aprendizaje en cualquier tiempo, en cualquier lugar, autodidacta y de forma interactiva.\\

El aprendizaje tradicional cara-a-cara tiene sus ventajas de ser familiar, cercano y confortable tanto para instructores como para estudiantes. Sin embargo, puede requerir viajar e interrumpir el trabajo, causando una pérdida de tiempo. En algunas situaciones, enviar al instructor al lugar puede ser impráctico. El e-Learning sirve como un mecanismo complementario para el aprendizaje remoto o permanente \cite{Zhang}.
%------------------------------------------------------------------------------------
\subsection{E-Learning asíncrono}
\label{sec:elearning}
El e-Learning asíncrono no requiere la participación simultánea de los aprendices e instructores. Se refiere a una situación de aprendizaje donde el evento de aprendizaje no toma lugar en tiempo real \cite{Zhang}. Las personas pueden aprender en cualquier tiempo. Por lo tanto, el e-Learning asíncrono es aprendizaje bajo demanda, lo cual les da a los aprendices más control sobre el proceso del aprendizaje y el contenido.\\

Por las características de este tipo de aprendizaje, se opta por tomar su concepto para ser utilizado en el presente trabajo terminal.
%------------------------------------------------------------------------------------
\subsection{Beneficios del E-Learning}

\begin{itemize}
\item Ahorros de costo y tiempo: Hasta un 40\% del dinero gastado en el aprendizaje individual, es abarcado por el costo del viaje \cite{Zhang}. Desde que los e-Learners no tienen que viajar hasta una locación específica, el e-Learning puede permitir un ahorro en los costos o en gastos indirectos.
\item Aprendizaje autodidacta: El e-Learning fomenta el aprendizaje autodidacta y autodirigido estructurando actividades centradas en el aprendiz. Cada aprendiz puede seleccionar las actividades de aprendizaje  que mejor se adecue a su entorno, intereses y su carrera en ese momento.
\item Ambiente de aprendizaje colaborativo: El e-Learning une a los aprendices y expertos separados físicamente para formar un aprendizaje colaborativo en línea \cite{Hiltz}.
\item Mejor acceso a los instructores: En un ambiente de e-Learning, los aprendices obtienen de los instructores ayuda y orientación en línea.
\end{itemize}

\clearpage