%------------------------------------------------------------------------------------
\section{Estado del arte}
Por medio de una investigación realizada, en la que se buscaron trabajos con propuestas similares a la nuestra, se encontró que dentro de ESCOM no existen trabajos que aborden nuestra propuesta, sin embargo se encuentran considerados algunos de ellos por contar con validaciones de movimientos; para el caso de investigaciones científicas, se tienen diferentes trabajos similares en el aspecto de que tienen el Karate Do como base y realizan una validación de movimientos de su técnica y apoyo al entrenamiento del mismo, así como de un trabajo en el que se realiza una validación de posiciones específicas que no son comúnmente detectadas en otros trabajos; y por último se consideraron algunos productos comerciales (videojuegos) que proponen una activación física al usuario y que por medio de una validación de movimientos reflejan un resultado de su desempeño en comparación con los movimientos programados dentro de la aplicación. Sin embargo, ningún trabajo encontrado plantea el concepto de un entrenamiento deportivo formal a distancia.\\

\begin{table}[H]
\centering
\begin{tabular}{| p{4 cm} | p{7 cm} | p{4 cm} |}
\hline
\multicolumn{3}{| >{\columncolor[rgb]{0.529412, 0.807843, 0.980392}} c |} {\textbf{Productos de investigación}}\\
\hline
\rowcolor[rgb]{0.529412, 0.807843, 0.980392} \hspace{3em}{\textbf{Software}} & \hspace{5em}{\textbf{Características}} & \hspace{2em}{\textbf{Publicación}}\\
\hline
Game based approach to learn Martial Art for beginners \cite{Chye} & Primera parte de una investigación en la cual se desarrolla un videojuego con el objetivo de que los usuarios aprendan un arte Marcial (Thai boxing). A través de puntos del cuerpo, utilizando algoritmos del análisis de la postura (distancia euclidiana).

En este artículo, se enfocan en la precisión de las mediciones de las partes del cuerpo humano (estimación de altura). & 2012 IEEE International Conference on Embedded and Real-Time Computing Systems and Applications.\\
\hline
In Search of a Usability of Kinect in the Training of Traditional Japanese ``KATA"\-Stylized Gestures and Movements \cite{Wada} & Proponen un sistema para el entrenamiento de ``KATAS", en este artículo se desarrolla una parte del sistema que obtiene coordenadas tridimensionales en 20 diferentes puntos de las articulaciones del cuerpo humano y es capaz de medir los ángulos en algunas posiciones. & IEEE 2013.\\
\hline
Computer Karate Trainer in Tasks of Personal and Homeland Security Defense \cite{Hachaj} & Se presenta una nueva posibilidad de utilizar lenguajes de descripción de gestos (GDL por sus siglas en inglés) para el reconocimiento de técnicas básicas de las artes marciales.  El enfoque del GDL permite no sólo analizar varias técnicas del Karate Shorin-Ryu sino que permite apoyar al entrenamiento y la enseñanza de dichas artes. & IFIP International Federation for Information Processing 2013.\\
\hline
Safe Lifting: An adaptive Training System for Factory Workers Using the Microsoft Kinect \cite{Delpresto} & Es un proyecto de investigación el cual tiene como objetivo el diseñar un sistema de monitoreo preciso que pueda ayudar a trabajadores de fábricas a corregir sus técnicas de levantamiento pesado por medio de recomendaciones de técnicas adaptativas. Apoyándose en el uso de la cámara de profundidad del sensor Kinect y definiendo las técnicas de levantamiento usando varias ecuaciones de levantamiento y varios modelos biomecánicos. & Proceedings of the 2013 IEEE Systems and Information Engineering Design Symposium, University of Virginia 2013.\\
\hline
\end{tabular}
\caption{Productos de investigación}
\end{table}

\begin{table}[H]
\centering
\begin{tabular}{| p{4 cm} | p{8 cm}| p{3 cm} |}
\hline
\multicolumn{3}{| >{\columncolor[rgb]{0.529412, 0.807843, 0.980392}} c|} {\textbf{Trabajo Terminal}}\\
\hline
\rowcolor[rgb]{0.529412, 0.807843, 0.980392} \hspace{4em}{\textbf{Software}} & \hspace{5em}{\textbf{Características}} & {\textbf{Precio en el mercado}}\\
\hline
TT2013-A015 Sistema de apoyo para el tratamiento del sobrepeso y obesidad infantil utilizando el sensor Kinect. & Este proyecto consiste en desarrollar una herramienta que permita llevar un seguimiento y control de rutinas de ejercicios físicos para un infante. El sistema le asignará una serie de ejercicios que el usuario deberá realizar y por medio del sensor Kinect se captarán los movimientos que va ejecutando, asimismo podrá conocer el registro de las actividades físicas que ha estado realizando. & No aplica\\
\hline
TT2012-B007 Control de movimientos de un robot vía Kinect & Sistema de transferencia de movimientos o habilidades, mediante una interfaz de natural de usuario (NUI, por sus siglas en inglés) con ayuda del sensor Kinect, la cual permite analizar el movimiento del cuerpo, y de esta manera dar instrucciones precisas al robot para que imite un movimiento. & No aplica\\
\hline
\multicolumn{3}{| >{\columncolor[rgb]{0.529412, 0.807843, 0.980392}} c|} {\textbf{Productos comerciales}}\\
\hline
Nike+ Kinect Training & Sea cual sea tu nivel, sea cual sea tu objetivo, con Nike+ Kinect Training podrás tener un entrenamiento personal en casa. Usando retroalimentación en tiempo real como en una sesión profesional, podrás elegir un programa que irá evolucionando según tus progresos. Esto es Nike+ Kinect Training para Xbox 360. & \$499\\
\hline
Dance central 3 & El juego consiste en representar los movimientos de baile que el personaje en pantalla va realizando. A la derecha tenemos los movimientos que estamos realizando, su nombre, y los que vendrán a continuación. Dance Central está pensado para aprender a bailar y representar los movimientos, muchos de ellos complicados de encadenar al principio, a la perfección. & \$299\\
\hline
\end{tabular}
\caption{Trabajos Terminales y Productos comerciales}
\end{table}