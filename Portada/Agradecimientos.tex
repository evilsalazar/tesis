% 	CUARTA PÁGINA DE AGRADECIMIENTOS
%--------------------------------------------------------------------------------------------------------
\thispagestyle{empty}	% Borra los estilos que se puedan aplicar en el resto del documento
% \newgeometry{left=2cm,bottom=2cm,right=2cm,top=2cm}	%Define nuevamente los margenes para unas paginas especificas

\begin{center}
\textbf{\huge Agradecimientos}\\[1cm]
\end{center}
% 	Agradecimientos de Eduardo
%--------------------------------------------------------------------------------------------------------
{\fontfamily{ptm}\selectfont{
Primero que nada quiero agradecer a mi familia, a mis padres por haberme brindado todo el apoyo del mundo, y no sólo en esta etapa sino en toda mi vida escolar, sin ese apoyo llegar a donde me encuentro habría sido una tarea dificilísima; les agradezco todo lo que han hecho por mí, las preocupaciones cada vez que me desvelaba, el trabajo intenso para sustentar económicamente mis gastos, el esfuerzo porque no me faltara nada en mis necesidades como el hogar, vestimenta y alimento; apoyarme en todas las decisiones que he tomado, alentarme a ser el mejor. A mis hermanas quienes se han preocupado por mí y de igual manera me han apoyado en mis decisiones, y un agradecimiento especial de igual manera a mis abuelos y tíos.\\

Agradezco a nuestros directores del trabajo terminal David Ortega y a Roberto Zagal, por habernos aceptado como sus alumnos, por brindarnos su confianza, y por apoyarnos en cada una de las dudas que surgieron a través del desarrollo de este trabajo. \\

Quiero a gradecer a mis compañeras del trabajo terminal, Lucero Salazar y Valeria Canchola, ya que sin ellas no habríamos obtenido el resultado que tenemos; agradecerles porque, aunque hubieron peleas y conflictos, el trabajo siempre se estuvo realizando, y cada quien fue un elemento muy importante en el desarrollo, porque todos trabajamos arduamente, y lo repito, sin ellas no habríamos tenido este resultado.\\

Quiero agradecer de manera especial al profesor Gelacio Castillo, quien a pesar de no haberse involucrado en el presente trabajo, ha sido un profesor y una persona que nos ha apoyado bastante a lo largo de nuestra estancia en la escuela, quien nos alienta a ser mejores, a no quedarnos ni conformarnos con lo que tenemos, si no que busquemos mucho más.\\

Todo lo que he realizado en la escuela, durante estos cinco años, este año de trabajo terminal, ha sido un gran logro, en el que he puesto todo mi esfuerzo, todo mi empeño, inclusive mi alma, porque es uno de los objetivos que me había propuesto desde hacía varios años, este es un momento que me había planteado, el llegar a este día, el pertenecer a esta honorable escuela, el convertirme en un ingeniero, todo esto, es un gran logro personal, un logro que ahora es parte de mí y que les dedico a todas las personas que me apoyaron. Hoy termino esta etapa, pero no es la última, la vida no se detiene, así que como se dice por ahí, ``No vamos a parar".\\

}}
\begin{flushright}
{\fontfamily{pzc}\selectfont{\Large Eduardo Adolfo Arroyo López}}
\end{flushright}
\clearpage
\thispagestyle{empty}
\hphantom{1cm}
\pagebreak
% 	Agradecimientos de Lucero
%--------------------------------------------------------------------------------------------------------
{\fontfamily{ptm}\selectfont{
Antes que nada, me gustaría expresar mi más sincero agradecimiento a mi familia, a mis padres porque confiaron en mis capacidades desde mi niñez, apoyándome en cada una de las grandes decisiones que he tomado en mi vida aun cuando parecieran las más locas decisiones. Es maravilloso ser la primera en alcanzar una ingeniería en mi familia por lo que me es grato saber que puedo ser un ejemplo para las futuras generaciones así como fueron mis hermanos y padres para mí.\\

Doy las gracias a mis compañeros de equipo de trabajo Eduardo y Valeria, por las discusiones estimulantes, para las noches de insomnio en que estábamos trabajando juntos antes de las entregas y por toda la diversión que hemos tenido no sólo en el último cuatro año, sino también en los años anteriores cuando en las clases donde coincidíamos también formábamos equipo. Fue la mejor decisión que pude tomar el trabajar a su lado.\\

También me encuentro muy agradecida con mis directores David y Roberto, quienes creyeron en nosotros desde un principio. En especial a David que sin haber sido sus alumnos, nos dio la oportunidad de demostrarle que podíamos realizar un trabajo profesional y de excelencia. Gracias por su paciencia, motivación, entusiasmo y conocimiento inmenso. Su guía nos ayudó en todo el tiempo de desarrollo de este gran trabajo. Yo no podría haber imaginado tener unos mejores directores y mentores para mi ingeniería.\\

De igual forma me gustaría darle las gracias a mi mejor amigo Sergio, que si bien ya no se encuentra entre nosotros, fue la primera persona que me impulso a estudiar una ingeniería en sistemas computaciones, siendo el primero en creer en mi talento y también mi primer gran mentor. Sé que te sentirías muy orgulloso de mí e igual que en aquella época me seguirías motivando para alcanzar otro sueño más.\\

Por último, pero no menos importante, gracias a Dios que siempre le da luz a mi vida aún en los momentos más oscuros.\\[3cm]
}}
\begin{flushright}
{\fontfamily{pzc}\selectfont{\Large Evila Lucero Salazar Ortiz}}
\end{flushright}
\clearpage
\thispagestyle{empty}
\hphantom{1cm}
\pagebreak
% \vspace{2cm}
% \textbf{{\Huge Universidad de Córdoba}\\[0.5cm]

% \restoregeometry