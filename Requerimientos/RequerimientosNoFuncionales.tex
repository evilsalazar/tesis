\section{Requerimientos no funcionales}

A continuación se muestran los requerimientos no funcionales los cuales estan separados en Restricciones y en Propiedades, especificando para cada uno de ellos un identificador único con el cual se hará referencia a los mismos en el documento, el nombre del requerimiento y su descripción.

%--------------------------------------------------------------------------------------------------------
\subsection{Restricciones}

\textbf{\textcolor[rgb]{0, 0, 0.545098}{RNF01-R \hspace{2cm} Lenguaje de programación}}\\
\rule[3mm]{17cm}{0.1mm}
\label{rnf:RNF01-R}
\textbf{Descripción: } El lenguaje de programación será C\# utilizando el SDK oficial de Kinect. \\
\textbf{Versión: } 2. \\

\textbf{\textcolor[rgb]{0, 0, 0.545098}{RNF02-R \hspace{2cm} Sistema Operativo}}\\
\rule[3mm]{17cm}{0.1mm}
\label{rnf:RNF02-R}
\textbf{Descripción: } El sistema operativo para ejecutar la herramienta será Windows 7 debido a que el SDK de Kinect para Windows 1.8  liberado en Noviembre del 2013 esta desarrollado para Windows además de ser el sistema operativo más utilizado en México y en el mundo \cite{WindowsSeven}. \\
\textbf{Versión: } 2. \\

\textbf{\textcolor[rgb]{0, 0, 0.545098}{RNF03-R \hspace{2cm} Características del hardware}}\\
\rule[3mm]{17cm}{0.1mm}
\label{rnf:RNF03-R}
\textbf{Descripción: } El equipo donde se ejecute la herramienta deberá tener como mínimo las siguientes características de hardware: \cite{Webb}. 
\begin{enumerate}
	\item Procesador dual-core de 2.66 GHz.
	\item Tarjeta gráfica que soporte las capacidades de Microsoft Direct X 9.0c.
	\item 2.0 Gb RAM.
	\item Sensor Kinect para Xbox 360.
	\item Adaptador de corriente USB para Kinect.
\end{enumerate} 
\textbf{Versión: } 2. \\

%\textbf{RNF4-R} & Características del servidor & El equipo servidor deberá tener las siguientes características de hardware:
%\begin{enumerate}
%	\item Procesador Intel® Xeon® serie 5500 y 5600. Intel® Xeon® de seis núcleos y Intel® Xeon® de cuatro núcleos.
%	\item Chipset Intel® 5500.
%	\item Hasta 128 GB (8 ranuras DIMM): DDR3 de 1 GB/2 GB/4 GB/8 GB/16 GB. Hasta 1333 MHz.
%	\item Unidad de estado sólido SATA de 2,5", SAS (10.000) SAS de 3,5" (15.000, 10.000), SAS Nearline (7.200), SATA (7.200).
%\end{enumerate} & v1\\
%\hline

%--------------------------------------------------------------------------------------------------------

\subsection{Propiedades}

\textbf{\textcolor[rgb]{0, 0, 0.545098}{RNF01-P \hspace{2cm} Captura de movimientos}}\\
\rule[3mm]{17cm}{0.1mm}
\label{rnf:RNF01-P}
\textbf{Descripción: } La captura de movimientos se hará con ayuda del sensor Kinect v1. \\
\textbf{Versión: } 2. \\

\textbf{\textcolor[rgb]{0, 0, 0.545098}{RNF02-P \hspace{2cm} Interfaz de herramienta}}\\
\rule[3mm]{17cm}{0.1mm}
\label{rnf:RNF02-P}
\textbf{Descripción: } La interacción con la herramienta con los usuarios será mediante una Interfaz Gráfica de Usuario, a excepción de las partes de captura y validación de movimientos. \\
\textbf{Versión: } 2. \\

\textbf{\textcolor[rgb]{0, 0, 0.545098}{RNF03-P \hspace{2cm} Mecanismo de comunicación}}\\
\rule[3mm]{17cm}{0.1mm}
\label{rnf:RNF03-P}
\textbf{Descripción: } El mecanismo de comunicación con el servidor que se implementará en la herramienta será por medio de un web service. \\
\textbf{Versión: } 2. \\

\textbf{\textcolor[rgb]{0, 0, 0.545098}{RNF04-P \hspace{2cm} Formato de archivos de ejercicios y movimientos}}\\
\rule[3mm]{17cm}{0.1mm}
\label{rnf:RNF04-P}
\textbf{Descripción: } El formato de archivos de la captura de movimientos será XML. \\
\textbf{Versión: } 2. \\

\clearpage