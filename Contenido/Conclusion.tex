\chapter{Conclusiones}

A lo largo del presente trabajo terminal  se logró desarrollar una herramienta con la cual se puede realizar la captura de movimientos de técnica de Karate Do así como la validación de los mismos, adicionalmente se introdujeron conceptos de ``entrenamiento a distancia", un tema que por su naturaleza de desarrollo se puede ubicar dentro de una parte del e-Learning, tema que aún es sujeto de investigación. Todo esto en conjunto fue realizado para cumplir con cada uno de los objetivos planteados al comienzo del desarrollo.\\

Se observó que en los últimos años han surgido alternativas de software para la realización de actividades físicas que incluyen o introducen conceptos lúdicos mientras que para deportes que requieren disciplina y un constante entrenamiento como el Karate Do, no existen dichas alternativas.\\

El trabajo realizado es una base para la captura y validación de movimientos específicos, de manera que se puede convertir en una importante contribución para aquellas áreas donde se requiera realizar/practicar movimientos precisos bajo la supervisión de personas expertas.\\

Se analizó que una de las aplicaciones para el algoritmo Dynamic Time Warping es para realizar la comparación de movimientos específicos del cuerpo humano.\\

Se concluyó que no se trabajaría con un estándar para cada movimiento porque en Karate Do no hay uno, pues existe el estilo libre, de manera que una de las principales características de la herramienta es que un entrenador experto es quién debe realizar la captura de movimientos.\\

También se observó que a pesar de que el sensor Kinect es una excelente opción para realizar la captura de movimientos, su versión 1 presenta problemas en la sobreposición de puntos durante la captura. Mismos que se pueden reducir al aplicar los smooth parameters que indica Microsoft pero que al final no se resuelven en su totalidad. Es importante puntualizar que existen limitantes en hardware que se eliminan parcialmente al aplicar diferentes técnicas pero teniendo en cuenta que el desarrollo de hardware aún continua, se liberan otros productos que como en el caso de Kinect 2 parecen ser mucho más prometedores.\\

Así mismo se establecieron los umbrales de validación mediante el Dynamic Time Warping para cada movimiento de técnica de cinta blanca de Karate Do.\\

Finalmente se concluye que hay mucho trabajo que realizar con respecto a la captura de movimientos para que la captura se realice cada vez de forma más precisa con respecto a los movimientos capturados; y que vale la pena indagar más en el concepto de entrenamiento a distancia, pues debido a la flexibilidad que existe en la captura, en un futuro puede ser posible que se adapte a diferentes actividades y tenga diferentes aplicaciones.\\

\chapter{Trabajo a futuro}

El trabajo desarrollado a lo largo del trabajo terminal contempló dos aspectos importantes: la captura y validación de movimientos así como el entrenamiento a distancia. Si bien el aspecto educacional no formó parte del alcance, sí se establece un nuevo tema de futuros desarrollos como lo establece el entrenamiento a distancia, mismo que es posible realizar a través de la validación de movimientos capturados en comparación con otros capturados por un entrenador experto en una disciplina, por lo tanto en futuras investigaciones y desarrollos sería interesante tratar lo siguiente:\\

Realizar un módulo de calibración para permitir la captura de movimientos más precisos. Para esta calibración se podría trabajar para poner plantillas específicas de un movimiento definidas a partir de los movimientos del entrenador experto, para que durante la captura se valide si se está acercando a dicho movimiento guardado en la plantilla.\\

Convertir la herramienta en una aplicación de e-Learning, es decir, de un aprendizaje completo del Karate Do.\\

Incluir más elementos administrativos en el módulo entrenador.\\

Realizar la captura de los movimientos de técnica de todas las cintas en Karate Do.\\

Incluir más elementos a la interfaz natural de usuario para realizar una captura general o dinámica, para capturar diversos movimientos que no se encuentren en el catálogo de movimientos.\\

Realizar una prueba comparativa del desempeño/rendimiento del algoritmo DTW con respecto con otros similares. \\

Realizar una base de datos de los movimientos capturados, para un uso posterior en alguna otra aplicación específica (similar a las bases de datos de rostros, huellas digitales y demás datos usados usualmente en reconocimiento de patrones).\\