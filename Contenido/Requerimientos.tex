\chapter{Análisis de requerimientos}
En el presente capítulo se realiza el análisis correspondiente para la obtención de los requerimientos de la herramienta, comenzando por la definición de los actores participantes y el alcance planteado, obteniendo así el listado de los requerimientos y las reglas de negocio.
%------------------------------------------------------------------------------------
\section{Definición de actores}
En la presente sección se realiza la especificación de los actores que tendrán interacción con la Herramienta de Apoyo al Entrenamiento de Karate Do. A continuación se describen los actores de acuerdo a las actividades que desarrollan.
%------------------------------------------------------------------------------------
\subsection{Entrenador}
\label{act:Entrenador} 
\textbf{\textcolor[rgb]{0, 0, 0.545098}{Nombre:}} Entrenador \\

\textbf{\textcolor[rgb]{0, 0, 0.545098}{Descripción:}} Persona experta en el arte marcial del Karate Do, la cual domina todos los movimientos de técnica y ejercicios de una rutina de entrenamiento, además de llevar un seguimiento del desempeño de cada uno de sus aprendices. \\

\textbf{\textcolor[rgb]{0, 0, 0.545098}{Responsabilidades:}}
\begin{itemize}
	\item Registrar a los practicantes en la herramienta.
	\item Registrar movimientos de técnica. 
	\item Registrar rutinas de entrenamiento.
	\item Visualizar la información de registro de los practicantes.
	\item Visualizar la lista de los ejercicios de calentamiento.
	\item Visualizar la lista de los movimientos de técnica.
	\item Visualizar la lista de las rutinas de entrenamiento.
	\item Visualizar la trayectoria del desempeño de los practicantes.
	\item Modificar la información de registro de los practicantes.
	\item Modificar los identificadores de las rutinas de entrenamiento.
	\item Asignar a los practicantes una rutina.
	\item Realizar comentarios sobre el desempeño de cada Practicante.
\end{itemize}
\vspace{1em}

\textbf{\textcolor[rgb]{0, 0, 0.545098}{Perfil:}}
\begin{itemize}
	\item Contar con un grado de al menos cinta negra primer Dan.
	\item Conocimientos en el uso de computadora.
	\item Conocimientos básicos en el uso del sensor Kinect.
\end{itemize}
\vspace{1em}

\textbf{\textcolor[rgb]{0, 0, 0.545098}{Cantidad:}} Uno.
%------------------------------------------------------------------------------------
\subsection{Practicante}
\label{act:Practicante}
\textbf{\textcolor[rgb]{0, 0, 0.545098}{Nombre:}} Practicante. \\

\textbf{\textcolor[rgb]{0, 0, 0.545098}{Descripción:}} Aprendiz de Karate Do cuyo principal interés es aprender nuevas técnicas o mejorar sus habilidades en dicho arte marcial, la cual entrena y realiza las rutinas de entrenamiento registradas por un Entrenador experto. \\

\textbf{\textcolor[rgb]{0, 0, 0.545098}{Responsabilidades:}}
\begin{itemize}
	\item Visualizar las rutinas de entrenamiento asignadas por el Entrenador.
	\item Visualizar los comentarios realizados por su Entrenador.
	\item Visualizar su información de registro.
	\item Modificar la contraseña de inicio de sesión en la herramienta.
	\item Replicar los ejercicios de calentamiento de una rutina seleccionada.
	\item Replicar los movimientos de técnica de una rutina seleccionada.
	\item Enviar la rutina de entrenamiento replicada.
\end{itemize}
\vspace{1em}

\clearpage

\textbf{\textcolor[rgb]{0, 0, 0.545098}{Perfil:}}
\begin{itemize}
	\item Gozar de buena salud y no padecer una enfermedad que se vea afectada por el esfuerzo físico.
	\item No contar con alguna discapacidad física que impida realizar los movimientos propuestos por la herramienta. 
	\item Conocimientos en el uso de una computadora.
	\item Conocimientos básicos en el uso del sensor Kinect.
	%*\item Estatura 
\end{itemize}
\vspace{1em}

\textbf{\textcolor[rgb]{0, 0, 0.545098}{Cantidad:}} De uno a diez, cada uno con una sesión independiente.
%------------------------------------------------------------------------------------
\section{Alcance}
\label{sec:Alcance}
\textbf{El alcance de esta herramienta contempla los siguientes puntos:}\\
\begin{enumerate}
	\item \textbf{\textcolor[rgb]{0, 0, 0.545098}{Un Entrenador.}}\\
	El presente trabajo terminal está planteado como una herramienta de apoyo al entrenamiento de Karate Do, es decir estamos transportando parte de una clase de entrenamiento, a un entrenamiento a distancia, y una clase normal consiste en un solo Entrenador dando instrucciones a un grupo de practicantes.\\
	\item \textbf{\textcolor[rgb]{0, 0, 0.545098}{De 1 a 10 practicantes, cada uno con una sesión independiente.}}\\
	En base a una encuesta realizada (véase anexo \ref{anex:encuesta} \nameref{anex:encuesta}), determinamos que lo ideal para un grupo de entrenamiento de Karate Do (una clase) es tener a 10 practicantes. Se propone de manera no simultánea ya que al ser una herramienta de apoyo, damos la libertad de que cada Practicante realice sus rutinas en el tiempo en que cada uno de ellos tenga la posibilidad, y no forzamos un horario determinado.\\
	\item \textbf{\textcolor[rgb]{0, 0, 0.545098}{Ejercicios de calentamiento y técnica.}}\\
	Un entrenamiento completo de Karate Do contempla ejercicios de calentamiento, ejercicios de resistencia y técnica (la cual a su vez se divide en movimientos, combate y katas); para el presente trabajo tomamos en cuenta sólo los ejercicios de calentamiento y los movimientos de técnica inicial ya que se trata de un apoyo al entrenamiento, pues los movimientos de técnica  requieren de un aprendizaje y un entrenamiento constante para mejorar esas habilidades, y para realizar los movimientos de técnica es necesario realizar un calentamiento previo, ya que aunque son movimientos, se requiere un esfuerzo físico e incluso de estiramiento de extremidades.\\
	\item \textbf{\textcolor[rgb]{0, 0, 0.545098}{10 ejercicios de calentamiento disponibles en total.}}\\
	Se proponen al menos 1 ejercicio de calentamiento por cada parte del cuerpo importante para realizar los movimientos de técnica y el uso o la selección de alguno de ellos dependerá del tipo de rutina que el Entrenador vaya a asignar.\\
	\item \textbf{\textcolor[rgb]{0, 0, 0.545098}{8 movimientos de técnica disponibles en total (2 ataques, 3 defensas y 3 posiciones).}}\\
	Se proponen los movimientos de técnica básicos correspondientes a la cinta blanca, el uso de cada uno de ellos dependerá del tipo de rutina que el Entrenador vaya a asignar.\\
	\item \textbf{\textcolor[rgb]{0, 0, 0.545098}{24 rutinas de entrenamiento registradas.}}\\
	De acuerdo a la encuesta realizada (véase \ref{anex:encuesta} \nameref{anex:encuesta}), determinamos que el número de clases ideal para el entrenamiento de Karate Do es de 3 por semana, en este trabajo se propone de esa manera 3 rutinas de entrenamiento por semana, y dado que el entrenamiento completo para poder realizar un examen y cambiar a cinta blanca es de 2 meses, se determina que el número total de rutinas que el Entrenador puede registrar como máximo es de 24.\\
	\item \textbf{\textcolor[rgb]{0, 0, 0.545098}{E-Learning asíncrono.}}\\
	Como se describe en el marco teórico en la sección \ref{sec:elearning} \nameref{sec:elearning}, el e-Learning asíncrono hace referencia a un aprendizaje en el que el profesor y el alumno no tienen que estar conectados al mismo tiempo, pero en el cual sigue existiendo un seguimiento y apoyo de dicho profesor hacia el alumno.\\
	\item \textbf{\textcolor[rgb]{0, 0, 0.545098}{Dos Interfaces Naturales de Usuario.}}\\
	Ambas aplicaciones contarán con una Interfaz Natural de UsuarioLa interfaz del Entrenador será una INU en las partes de de captura y validación de movimientos.\\
	

\end{enumerate}
\vspace{1em}
\textbf{La herramienta no contempla los siguientes puntos (posible trabajo a futuro):}\\

\begin{enumerate}
	\item \textbf{\textcolor[rgb]{0, 0, 0.545098}{Más de un Entrenador, es decir, una organización completa.}}\\
	Como se mencionó previamente, la herramienta está propuesta para apoyar el entrenamiento de una clase de Karate Do, por lo que el manejo de diferentes entrenadores, o una organización completa queda fuera de nuestros objetivos.\\
	\item \textbf{\textcolor[rgb]{0, 0, 0.545098}{Ejercicios de resistencia, entrenamiento de combate y katas.}}\\
	Los ejercicios de resistencia, son ejercicios que requieren de un esfuerzo físico mayor y que en diversas ocasiones no es necesario replicar algún movimiento de manera exacta, ya que para estos se requieren de otros factores como la condición física del Practicante o su complexión por mencionar algunos ejemplos.\\
El entrenamiento del combate requiere de otro Practicante para que se lleve a cabo de manera adecuada, además de un área de entrenamiento mayor ya que existen desplazamientos no uniformes cuando se realiza.\\
El entrenamiento de katas requiere de desplazamientos lineales horizontales y verticales por lo que estas validaciones quedan fuera de los objetivos principales.\\
	\item \textbf{\textcolor[rgb]{0, 0, 0.545098}{E-Learning síncrono.}}\\
	El e-Learning síncrono hace referencia al aprendizaje en el cual el profesor y el alumno se encuentran comunicados en tiempo real, como si se tratase de una clase en persona; como se mencionó previamente este trabajo es un apoyo y si eligiéramos este tipo de aprendizaje, tendríamos que restringir la realización de las rutinas a un horario determinado y entraría en conflicto con los objetivos y justificaciones propuestos.\\
	\item \textbf{\textcolor[rgb]{0, 0, 0.545098}{Uso de streaming.}}\\
	El uso de video en streaming requiere una cantidad significativa de ancho de banda de ambas partes de la herramienta para que la calidad de imagen sea favorable; en el presente trabajo se requiere que la herramienta funcione de manera adecuada sin necesidad de tener una conexión de alta velocidad.\\
	\item \textbf{\textcolor[rgb]{0, 0, 0.545098}{Grabación de video.}}\\
	La transferencia directa de archivos no está contemplada por el tipo de e-Learning, además de que la transferencia (subida y bajada) de archivos de video es costosa respecto al ancho de banda.\\
\end{enumerate}
\clearpage
%------------------------------------------------------------------------------------
\section{Requerimientos funcionales}

Los requerimientos funcionales se han abordado respecto a lo que se espera que la herramienta cubra en funcionalidad para el entrenamiento de la técnica inicial de cinta blanca del Karate Do. A continuación se muestra cada uno de los requerimientos identificados en el módulo Entrenador y en el módulo Practicante, especificando para cada uno de ellos un identificador único con el cual se hará referencia a los mismos en el documento, el nombre del requerimiento y su descripción. \\

%--------------------------------------------------------------------------------------------------------
\subsection{Requerimientos funcionales aplicación Entrenador}

\textbf{\textcolor[rgb]{0, 0, 0.545098}{RF01-E \hspace{2cm} Registro de practicantes}}\\
\rule[3mm]{17cm}{0.1mm}
\label{rf:RF01-E}
\textbf{Descripción: } Podrá realizar el registro de cada Practicante ingresando la información correspondiente al formulario de registro (ver \nameref{rn:RNR01} ). \\
\textbf{Versión: } 2. \\

\textbf{\textcolor[rgb]{0, 0, 0.545098}{RF02-E \hspace{2cm} Generación de contraseñas}}\\
\rule[3mm]{17cm}{0.1mm}
\label{rf:RF02-E}
\textbf{Descripción: } La contraseña del Practicante será generada automáticamente por la herramienta y será enviada al correo electrónico previamente registrado. \\
\textbf{Versión: } 2. \\

\textbf{\textcolor[rgb]{0, 0, 0.545098}{RF03-E \hspace{2cm} Consulta de información de los practicantes}}\\
\rule[3mm]{17cm}{0.1mm}
\label{rf:RF03-E}
\textbf{Descripción: } Podrá visualizar la lista de practicantes registrados además de su información personal. \\
\textbf{Versión: } 2. \\

\textbf{\textcolor[rgb]{0, 0, 0.545098}{RF04-E \hspace{2cm} Modificación de información de los practicantes}}\\
\rule[3mm]{17cm}{0.1mm}
\label{rf:RF04-E}
\textbf{Descripción: } Podrá realizar modificaciones en la información personal de los practicantes. \\
\textbf{Versión: } 2. \\

\textbf{\textcolor[rgb]{0, 0, 0.545098}{RF05-E \hspace{2cm} Consulta de trayectoria del Practicante}}\\
\rule[3mm]{17cm}{0.1mm}
\label{rf:RF05-E}
\textbf{Descripción: } Podrá visualizar el desempeño del Practicante. \\
\textbf{Versión: } 2. \\

\textbf{\textcolor[rgb]{0, 0, 0.545098}{RF06-E \hspace{2cm} Baja de Practicante}}\\
\rule[3mm]{17cm}{0.1mm}
\label{rf:RF06-E}
\textbf{Descripción: } Deberá tener la opción para poder dar de baja algún Practicante previamente registrado. \\
\textbf{Versión: } 2. \\

\textbf{\textcolor[rgb]{0, 0, 0.545098}{RF07-E \hspace{2cm} Captura de movimientos de técnica}}\\
\rule[3mm]{17cm}{0.1mm}
\label{rf:RF07-E}
\textbf{Descripción: } Podrá realizar la captura de movimientos de técnica. \\
\textbf{Versión: } 2. \\

\textbf{\textcolor[rgb]{0, 0, 0.545098}{RF08-E \hspace{2cm} Registro de rutinas de entrenamiento}}\\
\rule[3mm]{17cm}{0.1mm}
\label{rf:RF08-E}
\textbf{Descripción: } Podrá realizar el registro de rutinas de entrenamiento seleccionando los ejercicios de un catálogo y movimientos previamente capturados. \\
\textbf{Versión: } 2. \\

\textbf{\textcolor[rgb]{0, 0, 0.545098}{RF09-E \hspace{2cm} Modificación de identificadores de rutinas de entrenamiento}}\\
\rule[3mm]{17cm}{0.1mm}
\label{rf:RF09-E}
\textbf{Descripción: } Podrá realizar modificaciones en los identificadores de las rutinas de entrenamiento. \\
\textbf{Versión: } 2. \\

\textbf{\textcolor[rgb]{0, 0, 0.545098}{RF10-E \hspace{2cm} Baja de rutinas de entrenamiento}}\\
\rule[3mm]{17cm}{0.1mm}
\label{rf:RF10-E}
\textbf{Descripción: } Deberá tener la opción para poder dar de baja alguna rutina de entrenamiento. \\
\textbf{Versión: } 2. \\

\textbf{\textcolor[rgb]{0, 0, 0.545098}{RF11-E \hspace{2cm} Consulta del resultado del desempeño de las rutinas.}}\\
\rule[3mm]{17cm}{0.1mm}
\label{rf:RF11-E}
\textbf{Descripción: } Podrá visualizar el resultado del desempeño que la herramienta generará de forma porcentual con respecto a la exactitud de cada movimiento de técnica, en la rutina que el Practicante realizó. \\
\textbf{Versión: } 2. \\

\textbf{\textcolor[rgb]{0, 0, 0.545098}{RF12-E \hspace{2cm} Registro de comentarios de rutinas}}\\
\rule[3mm]{17cm}{0.1mm}
\label{rf:RF12-E}
\textbf{Descripción: } Podrá realizar comentarios escritos a la rutina realizada por el Practicante. \\
\textbf{Versión: } 2. \\

\textbf{\textcolor[rgb]{0, 0, 0.545098}{RF13-E \hspace{2cm} Modificación de comentarios}}\\
\rule[3mm]{17cm}{0.1mm}
\label{rf:RF13-E}
\textbf{Descripción: } Tendrá la opción para poder modificar los comentarios. \\
\textbf{Versión: } 2. \\
%--------------------------------------------------------------------------------------------------------

\subsection{Requerimientos funcionales aplicación Practicante}

\textbf{\textcolor[rgb]{0, 0, 0.545098}{RF01-P \hspace{2cm} Inicio de sesión del Practicante}}\\
\rule[3mm]{17cm}{0.1mm}
\label{rf:RF01-P}
\textbf{Descripción: } Podrá entrar a la herramienta, ingresando el nombre de usuario y contraseña previamente asignados. \\
\textbf{Versión: } 2. \\

\textbf{\textcolor[rgb]{0, 0, 0.545098}{RF02-P \hspace{2cm} Recuperar contraseña}}\\
\rule[3mm]{17cm}{0.1mm}
\label{rf:RF02-P}
\textbf{Descripción: } Podrá solicitar la recuperación de su contraseña. \\
\textbf{Versión: } 2. \\

\textbf{\textcolor[rgb]{0, 0, 0.545098}{RF03-P \hspace{2cm} Modificar contraseña}}\\
\rule[3mm]{17cm}{0.1mm}
\label{rf:RF03-P}
\textbf{Descripción: } Podrá modificar su contraseña actual. \\
\textbf{Versión: } 2. \\

\textbf{\textcolor[rgb]{0, 0, 0.545098}{RF04-P \hspace{2cm} Consulta de información del Practicante}}\\
\rule[3mm]{17cm}{0.1mm}
\label{rf:RF04-P}
\textbf{Descripción: } Podrá visualizar la información contenida en su registro. \\
\textbf{Versión: } 2. \\

\textbf{\textcolor[rgb]{0, 0, 0.545098}{RF05-P \hspace{2cm} Visualizar rutina de entrenamiento}}\\
\rule[3mm]{17cm}{0.1mm}
\label{rf:RF05-P}
\textbf{Descripción: } Podrá visualizar las rutinas asignadas por el Entrenador. \\
\textbf{Versión: } 2. \\

\textbf{\textcolor[rgb]{0, 0, 0.545098}{RF06-P \hspace{2cm} Seleccionar rutina de entrenamiento}}\\
\rule[3mm]{17cm}{0.1mm}
\label{rf:RF06-P}
\textbf{Descripción: } Podrá seleccionar la rutina de entrenamiento asignada por el Entrenador. \\
\textbf{Versión: } 2. \\

\textbf{\textcolor[rgb]{0, 0, 0.545098}{RF07-P \hspace{2cm} Realizar movimientos}}\\
\rule[3mm]{17cm}{0.1mm}
\label{rf:RF07-P}
\textbf{Descripción: } La herramienta identificará los movimientos que el Practicante ha replicado. \\
\textbf{Versión: } 2. \\

\textbf{\textcolor[rgb]{0, 0, 0.545098}{RF08-P \hspace{2cm} Repetir rutina de entrenamiento}}\\
\rule[3mm]{17cm}{0.1mm}
\label{rf:RF08-P}
\textbf{Descripción: } Deberá tener la opción de repetir la rutina de entrenamiento antes de enviarla al Entrenador. \\
\textbf{Versión: } 2. \\

\textbf{\textcolor[rgb]{0, 0, 0.545098}{RF09-P \hspace{2cm} Guardar rutina de entrenamiento}}\\
\rule[3mm]{17cm}{0.1mm}
\label{rf:RF09-P}
\textbf{Descripción: } Deberá poder guardar la rutina de entrenamiento realizada. \\
\textbf{Versión: } 2. \\

\textbf{\textcolor[rgb]{0, 0, 0.545098}{RF10-P \hspace{2cm} Enviar desempeño de rutina}}\\
\rule[3mm]{17cm}{0.1mm}
\label{rf:RF10-P}
\textbf{Descripción: } Deberá tener la opción para enviar el desempeño de la rutina de entrenamiento previamente realizada. \\
\textbf{Versión: } 2. \\

\textbf{\textcolor[rgb]{0, 0, 0.545098}{RF11-P \hspace{2cm} Consulta de comentarios del Entrenador}}\\
\rule[3mm]{17cm}{0.1mm}
\label{rf:RF11-P}
\textbf{Descripción: } Podrá visualizar los comentarios hechos por el Entrenador en cada rutina. \\
\textbf{Versión: } 2. \\

\textbf{\textcolor[rgb]{0, 0, 0.545098}{RF12-P \hspace{2cm} Consulta de desempeño}}\\
\rule[3mm]{17cm}{0.1mm}
\label{rf:RF12-P}
\textbf{Descripción: } Podrá visualizar el desempeño obtenido en un rutina realizada inmediatamente después de haberla finalizado. \\
\textbf{Versión: } 2. \\

\textbf{\textcolor[rgb]{0, 0, 0.545098}{RF13-P \hspace{2cm} Cierre de sesión}}\\
\rule[3mm]{17cm}{0.1mm}
\label{rf:RF13-P}
\textbf{Descripción: } Deberá tener la opción para poder salir de la herramienta.  \\
\textbf{Versión: } 2. \\

\clearpage
%--------------------------------------------------------------------------------------------------------

\section{Requerimientos no funcionales}

A continuación se muestran los requerimientos no funcionales los cuales estan separados en Restricciones y en Propiedades, especificando para cada uno de ellos un identificador único con el cual se hará referencia a los mismos en el documento, el nombre del requerimiento y su descripción.

%--------------------------------------------------------------------------------------------------------
\subsection{Restricciones}

\textbf{\textcolor[rgb]{0, 0, 0.545098}{RNF01-R \hspace{2cm} Lenguaje de programación}}\\
\rule[3mm]{17cm}{0.1mm}
\label{rnf:RNF01-R}
\textbf{Descripción: } El lenguaje de programación será C\# utilizando el SDK oficial de Kinect. \\
\textbf{Versión: } 2. \\

\textbf{\textcolor[rgb]{0, 0, 0.545098}{RNF02-R \hspace{2cm} Sistema Operativo}}\\
\rule[3mm]{17cm}{0.1mm}
\label{rnf:RNF02-R}
\textbf{Descripción: } El sistema operativo para ejecutar la herramienta será Windows 7 debido a que el SDK de Kinect para Windows 1.8  liberado en Noviembre del 2013 esta desarrollado para Windows además de ser el sistema operativo más utilizado en México y en el mundo \cite{WindowsSeven}. \\
\textbf{Versión: } 2. \\

\textbf{\textcolor[rgb]{0, 0, 0.545098}{RNF03-R \hspace{2cm} Características del hardware}}\\
\rule[3mm]{17cm}{0.1mm}
\label{rnf:RNF03-R}
\textbf{Descripción: } El equipo donde se ejecute la herramienta deberá tener como mínimo las siguientes características de hardware: \cite{Webb}. 
\begin{enumerate}
	\item Procesador dual-core de 2.66 GHz.
	\item Tarjeta gráfica que soporte las capacidades de Microsoft Direct X 9.0c.
	\item 2.0 Gb RAM.
	\item Sensor Kinect para Xbox 360.
	\item Adaptador de corriente USB para Kinect.
\end{enumerate} 
\textbf{Versión: } 2. \\

%\textbf{RNF4-R} & Características del servidor & El equipo servidor deberá tener las siguientes características de hardware:
%\begin{enumerate}
%	\item Procesador Intel® Xeon® serie 5500 y 5600. Intel® Xeon® de seis núcleos y Intel® Xeon® de cuatro núcleos.
%	\item Chipset Intel® 5500.
%	\item Hasta 128 GB (8 ranuras DIMM): DDR3 de 1 GB/2 GB/4 GB/8 GB/16 GB. Hasta 1333 MHz.
%	\item Unidad de estado sólido SATA de 2,5", SAS (10.000) SAS de 3,5" (15.000, 10.000), SAS Nearline (7.200), SATA (7.200).
%\end{enumerate} & v1\\
%\hline

%--------------------------------------------------------------------------------------------------------

\subsection{Propiedades}

\textbf{\textcolor[rgb]{0, 0, 0.545098}{RNF01-P \hspace{2cm} Captura de movimientos}}\\
\rule[3mm]{17cm}{0.1mm}
\label{rnf:RNF01-P}
\textbf{Descripción: } La captura de movimientos se hará con ayuda del sensor Kinect v1. \\
\textbf{Versión: } 2. \\

\textbf{\textcolor[rgb]{0, 0, 0.545098}{RNF02-P \hspace{2cm} Interfaz de herramienta}}\\
\rule[3mm]{17cm}{0.1mm}
\label{rnf:RNF02-P}
\textbf{Descripción: } La interacción con la herramienta con los usuarios será mediante una Interfaz Gráfica de Usuario, a excepción de las partes de captura y validación de movimientos. \\
\textbf{Versión: } 2. \\

\textbf{\textcolor[rgb]{0, 0, 0.545098}{RNF03-P \hspace{2cm} Mecanismo de comunicación}}\\
\rule[3mm]{17cm}{0.1mm}
\label{rnf:RNF03-P}
\textbf{Descripción: } El mecanismo de comunicación con el servidor que se implementará en la herramienta será por medio de un web service. \\
\textbf{Versión: } 2. \\

\textbf{\textcolor[rgb]{0, 0, 0.545098}{RNF04-P \hspace{2cm} Formato de archivos de ejercicios y movimientos}}\\
\rule[3mm]{17cm}{0.1mm}
\label{rnf:RNF04-P}
\textbf{Descripción: } El formato de archivos de la captura de movimientos será XML. \\
\textbf{Versión: } 2. \\

\clearpage
\section{Reglas de negocio} 

% Definiciones

\subsection{\normalsize{\textcolor[rgb]{0, 0, 0.545098}{RND01 Entrenamiento completo}}}
\label{rn:RND01}
\rule[3mm]{16.59cm}{0.1mm} \vspace{1mm}
\textbf{Tipo:} Definición.\\
\textbf{Descripción:} Un entrenamiento completo recomendado en la herramienta consiste en 3 rutinas semanales asignadas al Practicante.\\
\textbf{Referenciado por: } \nameref{cu:CUE04.3} \\

\subsection{\normalsize{\textcolor[rgb]{0, 0, 0.545098}{RND02 Conformación de rutina}}}
\label{rn:RND02}
\rule[3mm]{16.59cm}{0.1mm} \vspace{1mm}
\textbf{Tipo:} Definición.\\
\textbf{Descripción:} Una rutina se compone de:
\begin{itemize} \itemsep1pt \parskip0pt \parsep0pt
	\item Ejercicios de calentamiento.
	\item Movimientos de técnica.
\end{itemize}
\textbf{Referenciado por: } \nameref{cu:CUE03.1} \\

\subsection{\normalsize{\textcolor[rgb]{0, 0, 0.545098}{RND03 Indicadores de desempeño}}}
\label{rn:RND03}
\rule[3mm]{16.59cm}{0.1mm} \vspace{1mm}
\textbf{Tipo:} Definición.\\
\textbf{Descripción:} Un indicador de desempeño es una medida en términos de porcentaje, el cual es un promedio en la ejecución de los movimientos de una rutina realizada por el Practicante, con respecto a la información capturada por el Entrenador. El desempeño puede variar entre 0\% y 100\%, donde 0\% representa el desempeño más bajo posible de obtener y el 100\% representa el máximo.\\
Un indicador de desempeño particular de movimiento es el promedio de las repeticiones capturadas correctamente por cada movimiento.\\
Un indicador de desempeño general es el promedio de los indicadores de desempeño particulares de todos los movimientos de técnica de una rutina.\\
\textbf{Referenciado por: } \nameref{cu:CUE04.4}.\\


% Restricciones

\subsection{\normalsize{\textcolor[rgb]{0, 0, 0.545098}{RNR01 Información correcta}}}
\label{rn:RNR01}
\rule[3mm]{16.59cm}{0.1mm} \vspace{1mm}
\textbf{Tipo:} Restricción.\\
\textbf{Descripción:} Los datos proporcionados a la herramienta que son marcados como ``requeridos" no se deben
omitir. Todos los datos proporcionados a la herramienta deben respetar el formato y pertenecer al tipo
de dato especificado en el \nameref{sec:diccionario}; así como estar dentro de la longitud máxima o
mínima definida. \\
\textbf{Referenciado por: } \nameref{cu:CUE01}, \nameref{cu:CUE03.1}, \nameref{cu:CUE03.2}, \nameref{cu:CUE04.1}, \nameref{cu:CUE04.4.1}, \nameref{cu:CUP01}, \nameref{cu:CUP02.1}.\\

\subsection{\normalsize{\textcolor[rgb]{0, 0, 0.545098}{RNR02 Información de registro del Practicante}}}
\label{rn:RNR02}
\rule[3mm]{16.59cm}{0.1mm} \vspace{1mm}
\textbf{Tipo:} Restricción.\\
\textbf{Descripción:} Los datos del Practicante que el Entrenador registra en la herramienta son:
\begin{itemize} \itemsep1pt \parskip0pt \parsep0pt \itemsep1pt \parskip0pt \parsep0pt
	\item Nombre de usuario.
	\item Nombre(s).
	\item Apellido(s).
	\item Edad.
	\item Domicilio.
	\item Teléfono.
	\item Correo electrónico.
	\item Tipo de avatar
	\item Peso.
	\item Estatura.
	\item Grupo sanguíneo.
	\item Grado.
\end{itemize}
\textbf{Referenciado por: } \nameref{cu:CUE01} .\\

\subsection{\normalsize{\textcolor[rgb]{0, 0, 0.545098}{RNR03 Información obligatoria en registro de Practicantes}}}
\label{rn:RNR03}
\rule[3mm]{16.59cm}{0.1mm} \vspace{1mm}
\textbf{Tipo:} Restricción.\\
\textbf{Descripción:} Los datos obligatorios en el registro de Practicantes son los siguientes:
\begin{itemize} \itemsep1pt \parskip0pt \parsep0pt
	\item Nombre(s).
	\item Apellido(s).
	\item Correo electrónico.
	\item Nombre de usuario.
\end{itemize}
\textbf{Referenciado por: } \nameref{cu:CUE01}, \nameref{cu:CUE04.1} .\\

\subsection{\normalsize{\textcolor[rgb]{0, 0, 0.545098}{RNR04 Formato correcto de datos personales del Practicante}}}
\label{rn:RNR04}
\rule[3mm]{16.59cm}{0.1mm} \vspace{1mm}
\textbf{Tipo:} Restricción.\\
\textbf{Descripción:} Se describe a continuación el formato de información correcta de datos personales del Practicante:
\begin{itemize} \itemsep1pt \parskip0pt \parsep0pt
	\item Nombre(s). Una cadena entre 1 y 25 caracteres del alfabeto incluyendo acentos.
	\item Apellido(s). Una cadena entre 1 y 25 caracteres del alfabeto incluyendo acentos.
	\item Una cadena de máximo 100 caracteres de tipo alfanuméricos.
	\item Peso. Una cadena de máximo 7 caracteres de tipo numéricos, expresada en kilogramos. Puede seguir la siguiente estructura ordenada si se desean especificar gramos.
	\begin{itemize} \itemsep1pt \parskip0pt \parsep0pt
		\item Números.
		\item Carácter ``.".
		\item Números.
	\end{itemize}
	\textbf{Ejemplo:} 58.500
	\item Estatura. Una cadena de máximo 4 caracteres de tipo numéricos, expresada en metros. Puede seguir la siguiente estructura ordenada si se desean especificar centímetros.
	\begin{itemize} \itemsep1pt \parskip0pt \parsep0pt
		\item Números.
		\item Carácter ``.".
		\item Números.
	\end{itemize}
	\textbf{Ejemplo:} 1.60
	\item Edad. Una cadena de máximo 3 caracteres de tipo numéricos.
	\item Teléfono. Una cadena de máximo 15 caracteres de tipo numéricos. 
	\item Grupo sanguíneo. Los grupos sanguíneos disponibles para selección en la herramienta son:  Desconocido, AB+, AB-, A+, A-, B+, B-, O+,y O-.
	\item Grado de cinta. Los grados disponibles para selección en la herramienta son: Ninguna, Blanca, Blanca avanzada, Morada, Amarilla, Naranja, Verde, Azul, Café, Negra 1er Dan, Negra 2do Dan, Negra 3er Dan, Negra 4to Dan, Negra 5to Dan y Negra 6to Dan.
\end{itemize}
\textbf{Referenciado por: } \nameref{cu:CUE01}, \nameref{cu:CUE04.1} .\\

\subsection{\normalsize{\textcolor[rgb]{0, 0, 0.545098}{RNR05 Formato de nombre de usuario}}}
\label{rn:RNR05}
\rule[3mm]{16.59cm}{0.1mm} \vspace{1mm}
\textbf{Tipo:} Restricción.\\
\textbf{Descripción:} El nombre de usuario debe ser una cadena de entre 4 y 20 caracteres alfanuméricos. Se deben considerar las siguientes restricciones:
\begin{itemize} \itemsep1pt \parskip0pt \parsep0pt
	\item El nombre de usuario debe ser único.
	\item No deben existir espacios.
	\item Se puede hacer uso de guión bajo.
\end{itemize}
\textbf{Ejemplo:} Daniel\_San1\\
\textbf{Referenciado por:} \nameref{cu:CUE01} .\\

\subsection{\normalsize{\textcolor[rgb]{0, 0, 0.545098}{RNR06 Asignación de nombre de usuario del Practicante}}}
\label{rn:RNR06}
\rule[3mm]{16.59cm}{0.1mm} \vspace{1mm}
\textbf{Tipo:} Restricción.\\
\textbf{Descripción:} El Entrenador sólo puede asignar el nombre de usuario una vez por cada registro de un Practicante.\\
\textbf{Referenciado por: } \nameref{cu:CUE01} .\\

\subsection{\normalsize{\textcolor[rgb]{0, 0, 0.545098}{RNR07 Formato de correo electrónico}}}
\label{rn:RNR07}
\rule[3mm]{16.59cm}{0.1mm} \vspace{1mm}
\textbf{Tipo:} Restricción.\\
\textbf{Descripción:} El correo electrónico debe ser una cadena de máximo 50 caracteres con la siguiente estructura ordenada:
\begin{itemize} \itemsep1pt \parskip0pt \parsep0pt
	\item Cadena de caracteres.
	\item Caracter ``@".
	\item Cadena de caracteres.
	\item Caracter ``.".
	\item Cadena de caracteres.
\end{itemize}
\textbf{Ejemplo:} correo@dominio.com\\
\textbf{Referenciado por:} \nameref{cu:CUE01}, \nameref{cu:CUE04.1} .\\

\subsection{\normalsize{\textcolor[rgb]{0, 0, 0.545098}{RNR08 Formato de la contraseña generada por la herramienta}}}
\label{rn:RNR08}
\rule[3mm]{16.59cm}{0.1mm} \vspace{1mm}
\textbf{Tipo:} Restricción.\\
\textbf{Descripción:} La longitud de la contraseña debe ser entre 8 y 16 caracteres. Debe contener al menos una letra en mayúscula y al menos un dígito. 
Se deben considerar las siguientes restricciones: 
\begin{itemize} \itemsep1pt \parskip0pt \parsep0pt
	\item No debe haber caracteres especiales.
	\item No debe haber letras acentuadas.
	\item No deben existir espacios.
\end{itemize}
\textbf{Ejemplo:} sdsdEsd4\\
\textbf{Referenciado por:} \nameref{cu:CUE01} .\\

\subsection{\normalsize{\textcolor[rgb]{0, 0, 0.545098}{RNR09 Formato de la contraseña ingresada por el Practicante}}}
\label{rn:RNR09}
\rule[3mm]{16.59cm}{0.1mm} \vspace{1mm}
\textbf{Tipo:} Restricción.\\
\textbf{Descripción:} La longitud de la contraseña debe ser entre 8 y 16 caracteres. Debe contener al menos una letra en mayúscula y al menos un dígito. 
Se deberán considerar las siguientes restricciones: 
\begin{itemize} \itemsep1pt \parskip0pt \parsep0pt
	\item No debe haber caracteres especiales.
	\item No debe haber letras acentuadas.
	\item No deben existir espacios.
\end{itemize}
\textbf{Ejemplo:} Karatekid01\\
\textbf{Referenciado por:} \nameref{cu:CUP02.1} .\\

\subsection{\normalsize{\textcolor[rgb]{0, 0, 0.545098}{RNR10 Formato de fecha de inscripción}}}
\label{rn:RNR10}
\rule[3mm]{16.59cm}{0.1mm} \vspace{1mm}
\textbf{Tipo:} Restricción.\\
\textbf{Descripción:} La fecha de inscripción se determina por medio de la fecha del sistema operativo en el momento en que se realiza un registro. \\
\textbf{Referenciado por: } \nameref{cu:CUE01} .\\

\subsection{\normalsize{\textcolor[rgb]{0, 0, 0.545098}{RNR11 Envío de contraseña}}}
\label{rn:RNR11}
\rule[3mm]{16.59cm}{0.1mm} \vspace{1mm}
\textbf{Tipo:} Restricción.\\
\textbf{Descripción:} La herramienta envía una contraseña al correo electrónico del Practicante, generada aleatoriamente con respecto al formato especificado en la RNR07 Formato de la contraseña generada por la herramienta. \\
\textbf{Referenciado por: } \nameref{cu:CUE01} .\\

\subsection{\normalsize{\textcolor[rgb]{0, 0, 0.545098}{RNR12 Recuperación de contraseña}}}
\label{rn:RNR12}
\rule[3mm]{16.59cm}{0.1mm} \vspace{1mm}
\textbf{Tipo:} Restricción.\\
\textbf{Descripción:} Cuando el Practicante requiere recuperar su contraseña la herramienta envía la contraseña actual al correo electrónico registrado.\\
\textbf{Referenciado por: } \nameref{cu:CUP01.1} .\\

\subsection{\normalsize{\textcolor[rgb]{0, 0, 0.545098}{RNR13 Registro repetido/duplicado}}}
\label{rn:RNR13}
\rule[3mm]{16.59cm}{0.1mm} \vspace{1mm}
\textbf{Tipo:} Restricción.\\
\textbf{Descripción:} Un Practicante no puede estar registrado más de una vez dentro de la herramienta, esto se verifica con el/los nombre(s) y apellido(s) de dicho Practicante y el nombre de usuario.\\
\textbf{Referenciado por: } \nameref{cu:CUE01} .\\

\subsection{\normalsize{\textcolor[rgb]{0, 0, 0.545098}{RNR14 Información obligatoria en el registro de rutinas}}}
\label{rn:RNR14}
\rule[3mm]{16.59cm}{0.1mm} \vspace{1mm}
\textbf{Tipo:} Restricción.\\
\textbf{Descripción:} Los datos obligatorios en el registro de rutinas son los siguientes:
\begin{itemize} \itemsep1pt \parskip0pt \parsep0pt
	\item Nombre de la rutina.
	\item Ejercicios de calentamiento.
	\item Número de repeticiones de ejercicios de calentamiento.
	\item Movimientos de técnica.
	\item Número de repeticiones de movimientos de técnica.
\end{itemize}
\textbf{Referenciado por: } \nameref{cu:CUE03.1}, \nameref{cu:CUE03.2} .\\

\subsection{\normalsize{\textcolor[rgb]{0, 0, 0.545098}{RNR15 Formato correcto para el registro de rutinas.}}}
\label{rn:RNR15}
\rule[3mm]{16.59cm}{0.1mm} \vspace{1mm}
\textbf{Tipo:} Restricción.\\
\textbf{Descripción:} Se describe a continuación el formato correcto de la información para el registro de rutinas:
\begin{itemize} \itemsep1pt \parskip0pt \parsep0pt
	\item Nombre de la rutina.  Una cadena de entre 1 y 25 caracteres de tipo alfanuméricos.
	\item Se seleccionan entre 3 y 5 ejercicios de calentamiento.
	\item Se seleccionan entre 8 y 20 repeticiones por cada ejercicio de calentamiento.
	\item Se seleccionan entre 2 y 4  movimientos de técnica.
	\item Se seleccionan entre 8 y 20 repeticiones por cada movimiento de técnica.
	\item Imagen descriptiva. Se selecciona una de las imágenes disponibles en la herramienta, o bien, se selecciona la imagen por defecto.
\end{itemize}
\textbf{Referenciado por: } \nameref{cu:CUE03.1}, \nameref{cu:CUE03.2} .\\

\subsection{\normalsize{\textcolor[rgb]{0, 0, 0.545098}{RNR16 Orden de registro de rutina}}}
\label{rn:RNR16}
\rule[3mm]{16.59cm}{0.1mm} \vspace{1mm}
\textbf{Tipo:} Restricción.\\
\textbf{Descripción:} El orden de registro de una rutina debe ser: primero ejercicios de calentamiento y después movimientos de técnica.\\
\textbf{Referenciado por: } \nameref{cu:CUE03.1} .\\

\subsection{\normalsize{\textcolor[rgb]{0, 0, 0.545098}{RNR17 Registro máximo de rutinas}}}
\label{rn:RNR17}
\rule[3mm]{16.59cm}{0.1mm} \vspace{1mm}
\textbf{Tipo:} Restricción.\\
\textbf{Descripción:} El Entrenador puede registrar hasta 24 rutinas diferentes con la técnica inicial de cinta blanca.\\
\textbf{Referenciado por: } \nameref{cu:CUE03.1} .\\

\subsection{\normalsize{\textcolor[rgb]{0, 0, 0.545098}{RNR18 Número de rutinas por semana}}}
\label{rn:RNR18}
\rule[3mm]{16.59cm}{0.1mm} \vspace{1mm}
\textbf{Tipo:} Restricción.\\
\textbf{Descripción:} El Entrenador puede asignar máximo 3 rutinas por semana a cada Practicante. \\
\textbf{Referenciado por: } \nameref{cu:CUE04.3} .\\

\subsection{\normalsize{\textcolor[rgb]{0, 0, 0.545098}{RNR19 Día de asignación de rutina}}}
\label{rn:RNR19}
\rule[3mm]{16.59cm}{0.1mm} \vspace{1mm}
\textbf{Tipo:} Restricción.\\
\textbf{Descripción:} El día para asignar rutinas a los Practicantes es el día Lunes.\\
\textbf{Referenciado por: } \nameref{cu:CUE04.3} .\\

\subsection{\normalsize{\textcolor[rgb]{0, 0, 0.545098}{RNR20 Multiplicidad de rutinas asignadas}}}
\label{rn:RNR20}
\rule[3mm]{16.59cm}{0.1mm} \vspace{1mm}
\textbf{Tipo:} Restricción.\\
\textbf{Descripción:} El Entrenador puede asignar la misma rutina al mismo Practicante varias veces por semana.\\
\textbf{Referenciado por: } \nameref{cu:CUE04.3} .\\

\subsection{\normalsize{\textcolor[rgb]{0, 0, 0.545098}{RNR21 Orden de realización de rutinas}}}
\label{rn:RNR21}
\rule[3mm]{16.59cm}{0.1mm} \vspace{1mm}
\textbf{Tipo:} Restricción.\\
\textbf{Descripción:} El Practicante puede realizar las rutinas sin importar el orden en que fueron asignadas por el Entrenador.\\
\textbf{Referenciado por: } \nameref{cu:CUP03.1} .\\

\subsection{\normalsize{\textcolor[rgb]{0, 0, 0.545098}{RNR22 Rutinas por día}}}
\label{rn:RNR22}
\rule[3mm]{16.59cm}{0.1mm} \vspace{1mm}
\textbf{Tipo:} Restricción.\\
\textbf{Descripción:} El Practicante puede realizar más de una rutina diferente por día.\\
\textbf{Referenciado por: } \nameref{cu:CUP03.1} .\\

\subsection{\normalsize{\textcolor[rgb]{0, 0, 0.545098}{RNR23 Eliminación de Practicantes}}}
\label{rn:RNR23}
\rule[3mm]{16.59cm}{0.1mm} \vspace{1mm}
\textbf{Tipo:} Restricción.\\
\textbf{Descripción:} Al eliminar un Practicante se elimina toda la información asociada a él.\\
\textbf{Referenciado por: } \nameref{cu:CUE04.2} .\\

\subsection{\normalsize{\textcolor[rgb]{0, 0, 0.545098}{RNR24 Eliminación de rutinas}}}
\label{rn:RNR24}
\rule[3mm]{16.59cm}{0.1mm} \vspace{1mm}
\textbf{Tipo:} Restricción.\\
\textbf{Descripción:} Una rutina registrada no puede ser eliminada mientras se encuentre asignada a un Practicante.\\
\textbf{Referenciado por: } \nameref{cu:CUE03.3} .\\

\subsection{\normalsize{\textcolor[rgb]{0, 0, 0.545098}{RNR25 Formato de comentario}}}
\label{rn:RNR25}
\rule[3mm]{16.59cm}{0.1mm} \vspace{1mm}
\textbf{Tipo:} Restricción.\\
\textbf{Descripción:} El comentario debe ser una cadena de máximo 500 caracteres de tipo alfanuméricos.\\
\textbf{Referenciado por: } \nameref{cu:CUE04.4.1} .\\

\subsection{\normalsize{\textcolor[rgb]{0, 0, 0.545098}{RNR26 Registrar comentarios}}}
\label{rn:RNR26}
\rule[3mm]{16.59cm}{0.1mm} \vspace{1mm}
\textbf{Tipo:} Restricción.\\
\textbf{Descripción:} El Entrenador puede registrar un sólo comentario por cada rutina realizada por cada Practicante.\\
\textbf{Referenciado por: } \nameref{cu:CUE04.4.1} .\\

\subsection{\normalsize{\textcolor[rgb]{0, 0, 0.545098}{RNR27 Modificación de comentarios}}}
\label{rn:RNR27}
\rule[3mm]{16.59cm}{0.1mm} \vspace{1mm}
\textbf{Tipo:} Restricción.\\
\textbf{Descripción:} El Entrenador puede modificar comentarios previamente guardados.\\
\textbf{Referenciado por: } \nameref{cu:CUE04.4.1} .\\

\subsection{\normalsize{\textcolor[rgb]{0, 0, 0.545098}{RNR28 Envío de indicadores de desempeño}}}
\label{rn:RNR28}
\rule[3mm]{16.59cm}{0.1mm} \vspace{1mm}
\textbf{Tipo:} Restricción.\\
\textbf{Descripción:} El Practicante puede enviar los indicadores de desempeño de una rutina sólo si la ha realizado y guardado previamente.\\
\textbf{Referenciado por: } \nameref{cu:CUP03.2} .\\

\subsection{\normalsize{\textcolor[rgb]{0, 0, 0.545098}{RNR29 Repetición de rutina}}}
\label{rn:RNR29}
\rule[3mm]{16.59cm}{0.1mm} \vspace{1mm}
\textbf{Tipo:} Restricción.\\
\textbf{Descripción:} El Practicante puede repetir la rutina completa siempre y cuando, no se haya enviado el desempeño de la misma.\\
\textbf{Referenciado por: } \nameref{cu:CUP03.1} .\\

\subsection{\normalsize{\textcolor[rgb]{0, 0, 0.545098}{RNR30 Desempeño de rutina realizada}}}
\label{rn:RNR30}
\rule[3mm]{16.59cm}{0.1mm} \vspace{1mm}
\textbf{Tipo:} Restricción.\\
\textbf{Descripción:} La herramienta muestra un indicador de desempeño general al término de cada rutina realizada por el Practicante. La herramienta muestra al Entrenador un indicador desempeño por cada movimiento de técnica realizado en una rutina por el Practicante.\\
\textbf{Referenciado por: } \nameref{cu:CUP03.1} .\\

\subsection{\normalsize{\textcolor[rgb]{0, 0, 0.545098}{RNR31 Información obligatoria en la modificación de contraseña}}}
\label{rn:RNR31}
\rule[3mm]{16.59cm}{0.1mm} \vspace{1mm}
\textbf{Tipo:} Restricción.\\
\textbf{Descripción:} Los datos obligatorios en la modificación de contraseña son los siguientes:
\begin{itemize} \itemsep1pt \parskip0pt \parsep0pt
	\item Nueva contraseña.
	\item Confirmar contraseña.
\end{itemize}
\textbf{Referenciado por: } \nameref{cu:CUP02.1} .\\

\subsection{\normalsize{\textcolor[rgb]{0, 0, 0.545098}{RNR32 Información obligatoria en el inicio de sesión del Practicante}}}
\label{rn:RNR32}
\rule[3mm]{16.59cm}{0.1mm} \vspace{1mm}
\textbf{Tipo:} Restricción.\\
\textbf{Descripción:} Los datos obligatorios para iniciar sesión en la herramienta son los siguientes:
\begin{itemize} \itemsep1pt \parskip0pt \parsep0pt
	\item Nombre de usuario.
	\item Contraseña.
\end{itemize}
\textbf{Referenciado por: } \nameref{cu:CUP01} .\\

\subsection{\normalsize{\textcolor[rgb]{0, 0, 0.545098}{RNR33 Envío de indicadores de desempeño duplicados}}}
\label{rn:RNR33}
\rule[3mm]{16.59cm}{0.1mm} \vspace{1mm}
\textbf{Tipo:} Restricción.\\
\textbf{Descripción:} Los indicadores de desempeño pueden ser enviados al Entrenador solo una vez por cada rutina.\\
\textbf{Referenciado por: } \nameref{cu:CUP03.2} .\\

\subsection{\normalsize{\textcolor[rgb]{0, 0, 0.545098}{RNR34 Captura de movimientos}}}
\label{rn:RNR34}
\rule[3mm]{16.59cm}{0.1mm} \vspace{1mm}
\textbf{Tipo:} Restricción.\\
\textbf{Descripción:} El método de captura de movimientos debe funcionar con usuarios de diferente estatura.\\
\textbf{Referenciado por: } \nameref{cu:CUE02} .\\

\subsection{\normalsize{\textcolor[rgb]{0, 0, 0.545098}{RNR35 Modelo guía para la captura}}}
\label{rn:RNR35}
\rule[3mm]{16.59cm}{0.1mm} \vspace{1mm}
\textbf{Tipo:} Restricción.\\
\textbf{Descripción:} Durante la captura se muestra un modelo de esqueleto que muestra los movimientos del Entrenador.\\
\textbf{Referenciado por: } \nameref{cu:CUE02} .\\

\subsection{\normalsize{\textcolor[rgb]{0, 0, 0.545098}{RNR36 Inicio de la captura}}}
\label{rn:RNR36}
\rule[3mm]{16.59cm}{0.1mm} \vspace{1mm}
\textbf{Tipo:} Restricción.\\
\textbf{Descripción:} Existe un movimiento específico por cada movimiento de técnica disponible en el catálogo para iniciar la captura.\\
\textbf{Referenciado por: } \nameref{cu:CUE02} .\\

\subsection{\normalsize{\textcolor[rgb]{0, 0, 0.545098}{RNR37 Finalización de captura}}}
\label{rn:RNR37}
\rule[3mm]{16.59cm}{0.1mm} \vspace{1mm}
\textbf{Tipo:} Restricción.\\
\textbf{Descripción:} Existe un movimiento específico por cada movimiento de técnica disponible en el catálogo para finalizar la captura.\\
\textbf{Referenciado por: } \nameref{cu:CUE02} .\\

\subsection{\normalsize{\textcolor[rgb]{0, 0, 0.545098}{RNR38 Formato de archivo de la captura}}}
\label{rn:RNR38}
\rule[3mm]{16.59cm}{0.1mm} \vspace{1mm}
\textbf{Tipo:} Restricción.\\
\textbf{Descripción:} El formato de archivo de la captura de un movimiento es XML.\\
\textbf{Referenciado por: } \nameref{cu:CUE02} .\\

\subsection{\normalsize{\textcolor[rgb]{0, 0, 0.545098}{RNR39 Animaciones en la realización de rutinas}}}
\label{rn:RNR39}
\rule[3mm]{16.59cm}{0.1mm} \vspace{1mm}
\textbf{Tipo:} Restricción.\\
\textbf{Descripción:} Durante la realización de una rutina se muestra un modelo 3D que sirve como guía para el Practicante.\\
\textbf{Referenciado por: } \nameref{cu:CUP03.1} .\\

\subsection{\normalsize{\textcolor[rgb]{0, 0, 0.545098}{RNR40 Validación de movimientos}}}
\label{rn:RNR40}
\rule[3mm]{16.59cm}{0.1mm} \vspace{1mm}
\textbf{Tipo:} Restricción.\\
\textbf{Descripción:} La validación de movimientos se realiza a través del algoritmo DTW (ver sección \ref{sec:DTW} \nameref{sec:DTW}).\\
\textbf{Referenciado por: } \nameref{cu:CUP03.1} .\\

% \subsection{\normalsize{\textcolor[rgb]{0, 0, 0.545098}{}}}
% \label{rn:}
% \rule[3mm]{16.59cm}{0.1mm} \vspace{1mm}
% \textbf{Tipo:} \\
% \textbf{Descripción:} 
% \begin{itemize} \itemsep1pt \parskip0pt \parsep0pt
	% \item 
	% \item
% \end{itemize}
% \textbf{Ejemplo:} \\
% \textbf{Referenciado por:} \nameref{} \\

\clearpage