\subsubsection{CUE02 Capturar movimientos}
\label{cu:CUE02}

\textbf{\textcolor[rgb]{0, 0, 0.545098}{Resumen}} \\

Este caso de uso brinda al Entrenador la posibilidad de capturar un movimiento de técnica de los disponibles en el catálogo de movimientos, para su posterior uso al crear rutinas.\\

\textbf{\textcolor[rgb]{0, 0, 0.545098}{Descripción}}

\begin{table}[H]
\centering
\begin{tabular}{| l | p{12 cm} |}
\hline
\rowcolor[rgb]{0.529412, 0.807843, 0.980392} {\textbf{Caso de uso:}} & \hspace{7em}{\textbf{CUE02 Capturar movimientos}}\\
\hline
\textbf{Actor:} &  \nameref{act:Entrenador}. \\
\hline
\textbf{Objetivo:} & Permitir al Entrenador capturar movimientos y guardarlos en la herramienta.\\
\hline
\textbf{Entradas:} & Para llevar a cabo el registro del movimiento capturado, el Entrenador debe hacer uso de la pantalla \nameref{pant:IUE02} donde el Entrenador debe capturar lo siguiente:
		\begin{compactitem} 
			\setlength\itemsep{-0.25em}
			\item Movimiento. Se obtiene mediante el sensor Kinect.
		\end{compactitem} \\
\hline
\textbf{Salidas:} & \vspace{-2mm}	%Para quitar el espacio en blanco de la fila
					\begin{compactitem}
						\setlength\itemsep{-0.25em}
						\item Se muestra en la pantalla \nameref{pant:IUE02} el mensaje de notificación \nameref{msj:MSG01},  cuando se registre el movimiento de manera exitosa.
					\end{compactitem}\\
\hline
\textbf{Precondiciones:} & 	\vspace{-2mm}	%Para quitar el espacio en blanco de la fila
							\begin{compactitem}
								\setlength\itemsep{-0.25em}
								\item Ninguna.
							\end{compactitem}\\
\hline
\textbf{Postcondiciones:} & \vspace{-2mm}	%Para quitar el espacio en blanco de la fila
							\begin{compactitem}
								\setlength\itemsep{-0.25em}
								\item Se registra en la herramienta un movimiento de técnica.
								\item Se actualizará la información del catálogo de \textit{movimientos de técnica}.
							\end{compactitem}\\
							
\hline
\textbf{Reglas de negocio:} & \vspace{-2mm}	%Para quitar el espacio en blanco de la fila
							\begin{compactitem}
								\setlength\itemsep{-0.25em}
								\item \nameref{rn:RNR34}
								\item \nameref{rn:RNR35}
								\item \nameref{rn:RNR36}
								\item \nameref{rn:RNR37}
								\item \nameref{rn:RNR38}
							\end{compactitem}\\							
\hline
\textbf{Errores:} &	\vspace{-2mm}	%Para quitar el espacio en blanco de la fila
					\begin{compactitem}
						\setlength\itemsep{-0.25em}
						\item Se muestra en la pantalla \nameref{pant:IUE02} el mensaje de error \nameref{msj:MSG25} cuando no se pueda guardar el movimiento en la base de datos.
					\end{compactitem}\\
\hline
\textbf{Tipo:} & Primario.\\
\hline	
\end{tabular}
\caption{Resumen de atributos de CUE02 Capturar movimientos}
\label{tab:CUE02}
\end{table} 
\clearpage

%--------------------------------------------------------------------------------------------------------
\textbf{\textcolor[rgb]{0, 0, 0.545098}{Trayectorias del caso de uso}} \\
%\textbf{Trayectorias del caso de uso} \\

\textbf{\large{Trayectoria principal}}

\begin{enumerate}
	\item \includegraphics[width=15pt, height=10pt]{./Figuras/iconosCU/usuario.png} Solicita capturar un movimiento seleccionando la opción \textit{Capturar} del menú principal \nameref{menu:ME01}.	
	\item \includegraphics[width=15pt]{./Figuras/iconosCU/herramienta.png} Muestra el menú de la pantalla \nameref{pant:IUE02}.
	\item \includegraphics[width=15pt, height=10pt]{./Figuras/iconosCU/usuario.png} Selecciona el movimiento a capturar del catálogo disponible mediante el gesto \textit{Seleccionar}.
	\item \includegraphics[width=15pt]{./Figuras/iconosCU/herramienta.png} Muestra la pantalla de captura \nameref{pant:IUE02}.
	\item \includegraphics[width=15pt, height=10pt]{./Figuras/iconosCU/usuario.png} Realiza el saludo de Karate Do.
	\item \includegraphics[width=15pt, height=10pt]{./Figuras/iconosCU/usuario.png} Se coloca en posición de alerta.
	\item \includegraphics[width=15pt, height=10pt]{./Figuras/iconosCU/usuario.png} Realiza el movimiento seleccionado según lo especificado en la regla de negocio \nameref{rn:RNR34} con ayuda del modelo que especifica la regla de negocio \nameref{rn:RNR35}. \textbf{[Trayectoria A]}
	\item \includegraphics[width=15pt]{./Figuras/iconosCU/herramienta.png} Comienza la captura del movimiento por medio del sensor Kinect según lo especificado en la regla de negocio \nameref{rn:RNR36}.
	\item \includegraphics[width=15pt, height=10pt]{./Figuras/iconosCU/usuario.png} Finaliza la captura del movimiento según lo especificado en la regla de negocio \nameref{rn:RNR37}. \textbf{[Trayectoria B]}
	\item \includegraphics[width=15pt]{./Figuras/iconosCU/herramienta.png} Captura el movimiento en el archivo según lo especificado en la regla de negocio \nameref{rn:RNR38}
	\item \includegraphics[width=15pt]{./Figuras/iconosCU/herramienta.png} Guarda el movimiento en la base de datos. \textbf{[Trayectoria C]}
	\item \includegraphics[width=15pt]{./Figuras/iconosCU/herramienta.png} Muestra el mensaje \nameref{msj:MSG01}.
\end{enumerate}
	
- - - - \textit{Fin del caso de uso.} \\


%.....................................................
\textbf{\large{Trayectoria alternativa A:}}\\
\textbf{Condición: } El \textit{Practicante} desea regresar a la pantalla anterior sin realizar la captura del movimiento.

\begin{enumerate}
	\item \includegraphics[width=15pt, height=10pt]{./Figuras/iconosCU/usuario.png} Solicita cancelar la operación y regresar a la pantalla anterior por medio del gesto \textit{Regresar}.
	\item \includegraphics[width=15pt]{./Figuras/iconosCU/herramienta.png} Regresa al menú de la pantalla \nameref{pant:IUE02}.
	\item Continua en el paso 3 de la trayectoria principal.
\end{enumerate}

- - - - \textit{Fin de trayectoria.} \\

%.....................................................
\textbf{\large{Trayectoria alternativa B:}}\\
\textbf{Condición: } El \textit{Entrenador} desea repetir el movimiento capturado.

\begin{enumerate}
	\item \includegraphics[width=15pt, height=10pt]{./Figuras/iconosCU/usuario.png} Solicita volver a capturar el movimiento colocándose en posición de alerta.
	\item \includegraphics[width=15pt]{./Figuras/iconosCU/herramienta.png} Continúa en el paso 6 de la trayectoria principal.
\end{enumerate}

- - - - \textit{Fin de trayectoria.} \\

%.....................................................
\textbf{\large{Trayectoria alternativa C:}}\\
\textbf{Condición: } La conexión ha fallado al guardar el movimiento capturado en la base de datos.

\begin{enumerate}
	\item \includegraphics[width=15pt]{./Figuras/iconosCU/herramienta.png} Muestra en la pantalla \nameref{pant:IUE02} el mensaje de error \nameref{msj:MSG25} cuando no se pueda guardar el movimiento en la base de datos.
	\item Continua en el paso 3 de la trayectoria principal.
\end{enumerate}

- - - - \textit{Fin de trayectoria.} \\

%\newpage
\clearpage