\subsubsection{CUE04.3 Asignar rutinas}
\label{cu:CUE04.3}

\textbf{\textcolor[rgb]{0, 0, 0.545098}{Resumen}} \\

Este caso de uso brinda al Entrenador la posibilidad de asignar hasta 3 rutinas de entrenamiento semanalmente por cada Practicante registrado.\\

\textbf{\textcolor[rgb]{0, 0, 0.545098}{Descripción}}

\begin{table}[H]
\centering
\begin{tabular}{| c | p{12 cm} |}
\hline
\rowcolor[rgb]{0.529412, 0.807843, 0.980392} {\textbf{Caso de uso:}} & \hspace{7em}{\textbf{CUE04.3 Asignar rutinas}}\\
\hline
\textbf{Actor:} &  \nameref{act:Entrenador} \\
\hline
\textbf{Objetivo:} & Permitir al Entrenador asignar rutinas de entrenamiento a los Practicantes.\\
\hline
\textbf{Entradas:} & Para llevar a cabo la asignación de rutinas el Entrenador debe hacer uso de la pantalla \nameref{pant:IUE04.3} y debe introducir la siguiente información:
	\begin{compactitem} 
			\setlength\itemsep{-0.25em}
			\item Rutinas de entrenamiento. Se selecciona con el mouse.
	\end{compactitem} \\
\hline
\textbf{Salidas:} & \vspace{-2mm}	%Para quitar el espacio en blanco de la fila
					\begin{compactitem}
						\setlength\itemsep{-0.25em}
						\item Se mostrará en la pantalla \nameref{pant:IUE04.3} el mensaje de notificación \nameref{msj:MSG01} cuando las rutinas se hayan asignado de manera exitosa.
						\item Se mostrará el mensaje de confirmación \nameref{msj:MSG07} cuando el Entrenador asigne las rutinas.
					\end{compactitem}\\
\hline
\textbf{Precondiciones:} & \vspace{-2mm}	%Para quitar el espacio en blanco de la fila
							\begin{compactitem}
								\setlength\itemsep{-0.25em}
								\item Se requiere que existan elementos en el catálogo de \textit{rutinas}.
							\end{compactitem}\\
\hline
\textbf{Postcondiciones:} & \vspace{-2mm}	%Para quitar el espacio en blanco de la fila
							\begin{compactitem}
								%\setlength\itemsep{-0.25em}
								\item Se actualizará la información del catálogo de \textit{rutinas asignadas}.
							\end{compactitem}\\
\hline
\textbf{Reglas de negocio:} & \vspace{-2mm}	%Para quitar el espacio en blanco de la fila
							\begin{compactitem}
								%\setlength\itemsep{-0.25em}
								\item \nameref{rn:RND01}
								\item \nameref{rn:RNR18}
								\item \nameref{rn:RNR19}
								\item \nameref{rn:RNR20}
							\end{compactitem}\\
\hline
\textbf{Errores:} & \vspace{-2mm}	%Para quitar el espacio en blanco de la fila
					\begin{compactitem}
						\setlength\itemsep{-0.25em}
						\item Se muestra en la pantalla \nameref{pant:IUE04.3} el mensaje de error \nameref{msj:MSG17} cuando el Entrenador trate de asignar rutinas un día que no sea lunes.
					\end{compactitem}\\
\hline
\end{tabular}
\end{table} 

\clearpage

\begin{table}[H]
\centering
\begin{tabular}{| c | p{12 cm} |}						
\hline
				& \vspace{-2mm}	%Para quitar el espacio en blanco de la fila
					\begin{compactitem}
						\setlength\itemsep{-0.25em}
						% \item Se muestra en la pantalla \nameref{pant:IUE04.3} el mensaje de error \nameref{msj:MSG17} cuando el Entrenador trate de asignar rutinas un día que no sea lunes.
						\item  Se muestra en en la pantalla \nameref{pant:IUE04.3} el mensaje de error \nameref{msj:MSG24}, cuando cuando debido a un error de conexión no se muestren los elementos de forma correcta.
						\item Se muestra en la pantalla \nameref{pant:IUE04.3} el mensaje de error \nameref{msj:MSG18} cuando el Entrenador no seleccione el mínimo de rutinas.
						\item Se muestra en la pantalla \nameref{pant:IUE04} el mensaje de error \nameref{msj:MSG23} cuando el Entrenador quiera asignar nuevamente rutinas el mismo día.
					\end{compactitem}\\
\hline
\textbf{Tipo:} & Secundario, viene del \nameref{cu:CUE04}.\\
\hline	
\end{tabular}
\caption{Resumen de atributos de CUE04.3 Asignar rutinas}
\label{tabla:CUE043}
\end{table} 

%--------------------------------------------------------------------------------------------------------
\textbf{\textcolor[rgb]{0, 0, 0.545098}{Trayectorias del caso de uso}} \\
%\textbf{Trayectorias del caso de uso} \\

\textbf{\large{Trayectoria principal}}

\begin{enumerate}
	\item \includegraphics[width=15pt, height=10pt]{./Figuras/iconosCU/usuario.png} Selecciona el \textit{Practicante} al cuál se asignarán las rutinas mediante el uso de la pantalla \nameref{pant:IUE04}.
	\item \includegraphics[width=15pt, height=10pt]{./Figuras/iconosCU/usuario.png} Solicita realizar la asignación de rutinas al \textit{Practicante} seleccionando la opción \textit{Asignar rutina} del menú \nameref{menu:ME04}.
	\item \includegraphics[width=15pt]{./Figuras/iconosCU/herramienta.png} Verifica que el día de asignación se el correcto, según lo especificado en la regla de negocio \nameref{rn:RNR19}. \textbf{[Trayectoria A]}
	\item \includegraphics[width=15pt]{./Figuras/iconosCU/herramienta.png} Verifica que el el Entrenador no haya asignado rutinas previamente ese mismo día, según lo especificado en la regla de negocio \nameref{rn:RNR18}. \textbf{[Trayectoria B]}
	\item \includegraphics[width=15pt]{./Figuras/iconosCU/herramienta.png} Busca la información del catálogo lista de rutinas. \textbf{[Trayectoria C]}
	% \item \includegraphics[width=15pt]{./Figuras/iconosCU/herramienta.png} Verifica que el estado del \textit{Practicante} no sea \textit{inactivo}, según lo especificado en las reglas de negocio \nameref{rn:RND03} y \nameref{rn:RNR29}. \textbf{[Trayectoria B]}  
	\item \includegraphics[width=15pt]{./Figuras/iconosCU/herramienta.png} Solicita al \textit{Entrenador} realizar la selección de rutinas a asignar mediante el uso de la pantalla \nameref{pant:IUE04.3}. 
	\item \includegraphics[width=15pt, height=10pt]{./Figuras/iconosCU/usuario.png} Selecciona las nuevas rutinas a asignar de la lista de rutinas de entrenamiento registradas según lo especificado en la regla de negocio \nameref{rn:RND01} y \nameref{rn:RNR20}.
	\item \includegraphics[width=15pt, height=10pt]{./Figuras/iconosCU/usuario.png} Solicita asignar la(s) nueva(s) rutina(s) al \textit{Practicante} seleccionando la opción \textit{Guardar}. \textbf{[Trayectoria D]} 
	\item \includegraphics[width=15pt]{./Figuras/iconosCU/herramienta.png} Muestra el mensaje de confirmación \nameref{msj:MSG07}.
	\item \includegraphics[width=15pt, height=10pt]{./Figuras/iconosCU/usuario.png} Selecciona la opción \textit{Aceptar}.
	\item \includegraphics[width=15pt]{./Figuras/iconosCU/herramienta.png} Registra las nuevas rutinas asignadas al \textit{Practicante}.
	\item \includegraphics[width=15pt]{./Figuras/iconosCU/herramienta.png} Muestra el mensaje de notificación \nameref{msj:MSG01}.
	\item \includegraphics[width=15pt]{./Figuras/iconosCU/herramienta.png} Muestra la pantalla \nameref{pant:IUE04}. 
\end{enumerate}
	
- - - - \textit{Fin del caso de uso.} \\

%.....................................................
\textbf{\large{Trayectoria alternativa A:}}\\
\textbf{Condición: } El \textit{Entrenador} trata de asignar una rutina en un día que no es lunes.

\begin{enumerate}
	\item \includegraphics[width=15pt]{./Figuras/iconosCU/herramienta.png} Muestra en la pantalla \nameref{pant:IUE04} el mensaje de error \nameref{msj:MSG17}.
	\item Fin del caso de uso.
\end{enumerate}

%.....................................................
\textbf{\large{Trayectoria alternativa B:}}\\
\textbf{Condición: } El Entrenador ya ha realizado la asignación de rutinas previamente ese mismo día.

\begin{enumerate}
	\item \includegraphics[width=15pt]{./Figuras/iconosCU/herramienta.png} Muestra en la pantalla \nameref{pant:IUE04}, el mensaje de error \nameref{msj:MSG23}, indicando que la operación no puede continuar.
	\item Fin del caso de uso.
\end{enumerate}

- - - - \textit{Fin de trayectoria.} \\

%.....................................................
\textbf{\large{Trayectoria alternativa C:}}\\
\textbf{Condición: } Debido a un error de conexión, no se muestran los elementos de forma correcta.

\begin{enumerate}
	\item \includegraphics[width=15pt]{./Figuras/iconosCU/herramienta.png} Muestra en la pantalla \nameref{pant:IUE04}, el mensaje de error \nameref{msj:MSG24}, indicando que la operación no puede continuar de forma correcta.
	\item Fin del caso de uso.
\end{enumerate}

- - - - \textit{Fin de trayectoria.} \\

%.....................................................
\textbf{\large{Trayectoria alternativa D:}}\\
\textbf{Condición: } El \textit{Entrenador} desea regresar a la pantalla anterior.

\begin{enumerate}
	\item \includegraphics[width=15pt, height=10pt]{./Figuras/iconosCU/usuario.png} Solicita regresar a la pantalla anterior presionando el botón \textit{Regresar}.
	\item \includegraphics[width=15pt]{./Figuras/iconosCU/herramienta.png} Muestra la pantalla \nameref{pant:IUE04}.
	\item Fin del caso de uso.
\end{enumerate}

- - - - \textit{Fin de trayectoria.} \\

%\newpage
\clearpage