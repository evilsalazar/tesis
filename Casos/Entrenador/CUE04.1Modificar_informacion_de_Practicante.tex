\subsubsection{CUE04.1 Modificar información de Practicante}
\label{cu:CUE04.1}

\textbf{\textcolor[rgb]{0, 0, 0.545098}{Resumen}} \\

Este caso de uso brinda al Entrenador la posibilidad de modificar los datos personales de algún Practicante seleccionado.\\

\textbf{\textcolor[rgb]{0, 0, 0.545098}{Descripción}}

\begin{table}[H]
\centering
\begin{tabular}{| l | p{12 cm} |}
\hline
\rowcolor[rgb]{0.529412, 0.807843, 0.980392} {\textbf{Caso de uso:}} & \hspace{4em}{\textbf{CUE04.1 Modificar información de Practicante}}\\
\hline
\textbf{Actor:} &  \nameref{act:Entrenador}. \\
\hline
\textbf{Objetivo:} & Permitir al Entrenador modificar los datos de un Practicante.\\
\hline
\textbf{Entradas:} & Para llevar a cabo la modificación de alguno de los datos del Practicante, el Entrenador debe hacer uso de la pantalla \nameref{pant:IUE04.1} y debe capturar la siguiente información:
		\begin{compactitem} 
			\setlength\itemsep{-0.25em}
			\item Nombre(s)del usuario. Se escriben desde el teclado.
			\item Apellido(s). Se escriben desde el teclado.
			\item Edad. Se escribe desde el teclado.
			\item Domicilio. Se escribe desde el teclado.
			\item Teléfono. Se escribe desde el teclado.
			\item Correo electrónico. Se escribe desde el teclado.
			\item Avatar del Practicante. Se selecciona con el mouse.
			\item Peso. Se escribe desde el teclado.
			\item Estatura. Se escribe desde el teclado.
			\item Grupo sanguíneo. Se selecciona con el mouse.
			\item Grado de cinta. Se selecciona con el mouse.
		\end{compactitem} \\
\hline
\textbf{Salidas:} & \vspace{-2mm}	%Para quitar el espacio en blanco de la fila
					\begin{compactitem}
						\setlength\itemsep{-0.25em}
							\item Se muestra en la pantalla \nameref{pant:IUE04.1} el mensaje de notificación \nameref{msj:MSG01}, cuando se modifiquen los datos del Practicante de manera exitosa.
						\end{compactitem}\\
\hline
\textbf{Precondiciones:} & 	\vspace{-2mm}	%Para quitar el espacio en blanco de la fila
							\begin{compactitem}
								\setlength\itemsep{-0.25em}
								\item Ninguna.
							\end{compactitem}\\
\hline
\textbf{Postcondiciones:} & \vspace{-2mm}	%Para quitar el espacio en blanco de la fila
							\begin{compactitem}
								%\setlength\itemsep{-0.25em}
								\item Se actualizará el catálogo: \textit{Practicantes} con la nueva información.
							\end{compactitem}\\
							
\hline
\textbf{Reglas de negocio:} & \vspace{-2mm}	%Para quitar el espacio en blanco de la fila
							\begin{compactitem}
								%\setlength\itemsep{-0.25em}
								\item \nameref{rn:RNR01}.
								\item \nameref{rn:RNR03}.
								\item \nameref{rn:RNR04}.
								\item \nameref{rn:RNR07}.
							\end{compactitem}\\							
\hline
\end{tabular}
\end{table} 

\clearpage

\begin{table}[H]
\centering
\begin{tabular}{| c | p{14 cm} |}
\hline
\textbf{Errores:} &	\vspace{-2mm}	%Para quitar el espacio en blanco de la fila
					\begin{compactitem}
						\setlength\itemsep{-0.25em}
						\item Se muestra en la pantalla \nameref{pant:IUE04.1} el mensaje de error \nameref{msj:MSG12} cuando el Entrenador no haya ingresado datos a los campos obligatorios.
						\item Se muestra en pantalla el mensaje de error \nameref{msj:MSG13} cuando el Entrenador haya ingresado datos con un formato incorrecto en algún campo.
					\end{compactitem}\\
\hline
\textbf{Tipo:} & Secundario, viene del \nameref{cu:CUE04}.\\
\hline	
\end{tabular}
\caption{Resumen de atributos de CUE04.1 Modificar información de Practicante}
\label{tab:CUE04.1}
\end{table} 

%--------------------------------------------------------------------------------------------------------
\textbf{\textcolor[rgb]{0, 0, 0.545098}{Trayectorias del caso de uso}} \\
%\textbf{Trayectorias del caso de uso} \\

\textbf{\large{Trayectoria principal}}

\begin{enumerate}
	\item \includegraphics[width=15pt, height=10pt]{./Figuras/iconosCU/usuario.png} Solicita modificar información de alguno de los Practicantes enlistados seleccionando el elemento deseado de la pantalla \nameref{pant:IUE04}.
	\item \includegraphics[width=15pt]{./Figuras/iconosCU/herramienta.png} Muestra el menú \nameref{menu:ME04}.
	\item \includegraphics[width=15pt, height=10pt]{./Figuras/iconosCU/usuario.png} Selecciona la opción \textit{Modificar Información}.
	\item \includegraphics[width=15pt]{./Figuras/iconosCU/herramienta.png} Muestra la información del Practicante previamente registrada.
	\item \includegraphics[width=15pt]{./Figuras/iconosCU/herramienta.png} Solicita al \textit{Entrenador} ingresar la información requerida para modificar la información del \textit{Practicante} mediante el uso de la pantalla \nameref{pant:IUE04.1}.
	\item \includegraphics[width=15pt, height=10pt]{./Figuras/iconosCU/usuario.png} Ingresa los datos nuevos del \textit{Practicante}.
	\item \includegraphics[width=15pt]{./Figuras/iconosCU/herramienta.png} Verifica que se hayan ingresado los valores en los campos obligatorios, según lo especificado en las reglas de negocio \nameref{rn:RNR01} y  \nameref{rn:RNR03}. \textbf{[Trayectoria A]}
	\item \includegraphics[width=15pt]{./Figuras/iconosCU/herramienta.png} Verifica que los valores en los campos cumplan con el formato correcto según lo especificado en las reglas de negocio \nameref{rn:RNR04} y \nameref{rn:RNR07} . \textbf{[Trayectoria B]}
	\item \includegraphics[width=15pt, height=10pt]{./Figuras/iconosCU/usuario.png} Solicita guardar los nuevos datos del \textit{Practicante} seleccionando la opción \textit{Guardar}. \textbf{[Trayectoria C]}
	\item \includegraphics[width=15pt]{./Figuras/iconosCU/herramienta.png} Actualiza los datos del \textit{Practicante}.
	\item \includegraphics[width=15pt]{./Figuras/iconosCU/herramienta.png} Muestra el mensaje de notificación \nameref{msj:MSG01}.
	\item \includegraphics[width=15pt]{./Figuras/iconosCU/herramienta.png} Regresa a la pantalla \nameref{pant:IUE04}.
\end{enumerate}
	
- - - - \textit{Fin del caso de uso.} \\

%.....................................................
\textbf{\large{Trayectoria alternativa A:}}\\
\textbf{Condición: } El \textit{Entrenador} no ingresó los valores correspondientes en los campos marcados como obligatorios.

\begin{enumerate}
	\item \includegraphics[width=15pt]{./Figuras/iconosCU/herramienta.png} Muestra en la pantalla \nameref{pant:IUE04.1}, el mensaje de error \nameref{msj:MSG12}.
	\item \includegraphics[width=15pt]{./Figuras/iconosCU/herramienta.png} Continúa en el paso 5 de la trayectoria principal.
\end{enumerate}

- - - - \textit{Fin de trayectoria.} \\

%.....................................................
\textbf{\large{Trayectoria alternativa B:}}\\
\textbf{Condición: } El \textit{Entrenador} ingresó valores con un formato incorrecto.

\begin{enumerate}
	\item \includegraphics[width=15pt]{./Figuras/iconosCU/herramienta.png} Muestra en pantalla el mensaje de error \nameref{msj:MSG13}.
	\item \includegraphics[width=15pt]{./Figuras/iconosCU/herramienta.png} Continúa en el paso 5 de la trayectoria principal.
\end{enumerate}

- - - - \textit{Fin de trayectoria.} \\

\textbf{\large{Trayectoria alternativa C:}}\\
\textbf{Condición: } El \textit{Entrenador} desea cancelar la operación.

\begin{enumerate}
	\item \includegraphics[width=15pt, height=10pt]{./Figuras/iconosCU/usuario.png} Solicita cancelar la operación seleccionando el botón \textit{Cancelar}.
	\item \includegraphics[width=15pt]{./Figuras/iconosCU/herramienta.png} Cancela la operación y regresa a la pantalla \nameref{pant:IUE04}.
	\item Fin del caso de uso.
\end{enumerate}

- - - - \textit{Fin de trayectoria.} \\

%.....................................................


%\newpage
\clearpage