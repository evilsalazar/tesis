\subsubsection{CUE03.3 Dar de baja rutinas}
\label{cu:CUE03.3}

\textbf{\textcolor[rgb]{0, 0, 0.545098}{Resumen}} \\

Este caso de uso brinda al Entrenador la posibilidad de eliminar alguna de las rutinas de entrenamiento previamente registradas.\\

\textbf{\textcolor[rgb]{0, 0, 0.545098}{Descripción}}

\begin{table}[H]
\centering
\begin{tabular}{| l | p{12 cm} |}
\hline
\rowcolor[rgb]{0.529412, 0.807843, 0.980392} {\textbf{Caso de uso:}} & \hspace{7em}{\textbf{CUE03.3 Dar de baja rutinas}}\\
\hline
\textbf{Actor:} &  \nameref{act:Entrenador}. \\
\hline
\textbf{Objetivo:} & Permitir al Entrenador eliminar alguna rutina de entrenamiento.\\
\hline
\textbf{Entradas:} & Para realizar alguna eliminación, el Entrenador debe hacer uso del menú \nameref{menu:ME03} en la pantalla \nameref{pant:IUE03}. \\
\hline
\textbf{Salidas:} & \vspace{-2mm}	%Para quitar el espacio en blanco de la fila
					\begin{compactitem}
						\setlength\itemsep{-0.25em}
						\item Se muestra en la pantalla \nameref{pant:IUE03}, el mensaje de notificación \nameref{msj:MSG01}, cuando se realice una eliminación exitosa.
					\end{compactitem}\\
\hline
\textbf{Precondiciones:} & 	\vspace{-2mm}	%Para quitar el espacio en blanco de la fila
							\begin{compactitem}
								\setlength\itemsep{-0.25em}
								\item Deben existir elementos en los catálogos de \textit{rutinas}.
							\end{compactitem}\\
\hline
\textbf{Postcondiciones:} & \vspace{-2mm}	%Para quitar el espacio en blanco de la fila
							\begin{compactitem}
								%\setlength\itemsep{-0.25em}
								\item Se actualizará el catálogo de rutinas eliminando el elemento seleccionado.
							\end{compactitem}\\
							
\hline
\textbf{Reglas de negocio:} & \vspace{-2mm}	%Para quitar el espacio en blanco de la fila
							\begin{compactitem}
								%\setlength\itemsep{-0.25em}
								\item \nameref{rn:RNR24}.
							\end{compactitem}\\							
\hline
\textbf{Errores:} &	\vspace{-2mm}	%Para quitar el espacio en blanco de la fila
					\begin{compactitem}
						\setlength\itemsep{-0.25em}
						\item Se muestra en la pantalla \nameref{pant:IUE03} el mensaje de error \nameref{msj:MSG09}, cuando el Entrenador desee eliminar una rutina que se encuentre asignada a un Practicante.
					\end{compactitem}\\
\hline
\textbf{Tipo:} & Secundario, viene del \nameref{cu:CUE03}.\\
\hline	
\end{tabular}
\caption{Resumen de atributos de CUE03.3 Dar de baja rutinas}
\label{tab:CUE033}
\end{table} 

%--------------------------------------------------------------------------------------------------------
\textbf{\textcolor[rgb]{0, 0, 0.545098}{Trayectorias del caso de uso}} \\
%\textbf{Trayectorias del caso de uso} \\

\textbf{\large{Trayectoria principal}}

\begin{enumerate}
	\item \includegraphics[width=15pt, height=10pt]{./Figuras/iconosCU/usuario.png} Solicita eliminar alguno de los elementos enlistados seleccionando el elemento deseado de la pantalla \nameref{pant:IUE03}.
	\item \includegraphics[width=15pt]{./Figuras/iconosCU/herramienta.png} Muestra el menú \nameref{menu:ME03}.
	\item \includegraphics[width=15pt, height=10pt]{./Figuras/iconosCU/usuario.png} Selecciona la opción \textit{Eliminar}.
	\item \includegraphics[width=15pt]{./Figuras/iconosCU/herramienta.png} Muestra el mensaje de confirmación \nameref{msj:MSG04}.
	\item \includegraphics[width=15pt, height=10pt]{./Figuras/iconosCU/usuario.png} Selecciona la opción \textit{Aceptar}. \textbf{[Trayectoria A]} 
	\item \includegraphics[width=15pt]{./Figuras/iconosCU/herramienta.png} Verifica que el elemento seleccionado no se encuentre asignado a un \textit{Practicante}, según lo especificado en la regla de negocio \nameref{rn:RNR24}. \textbf{[Trayectoria B]}
	\item \includegraphics[width=15pt]{./Figuras/iconosCU/herramienta.png} Muestra el mensaje de notificación \nameref{msj:MSG01} y regresa a la pantalla \nameref{pant:IUE03}.
\end{enumerate}
	
- - - - \textit{Fin del caso de uso.} \\

%.....................................................
\textbf{\large{Trayectoria alternativa A:}}\\
\textbf{Condición: } El \textit{Entrenador} desea cancelar la operación.

\begin{enumerate}
	\item \includegraphics[width=15pt, height=10pt]{./Figuras/iconosCU/usuario.png} Solicita cancelar la operación seleccionando la opción \textit{Cancelar}.
	\item \includegraphics[width=15pt]{./Figuras/iconosCU/herramienta.png} Cancela la operación y regresa a la pantalla \nameref{pant:IUE03}.
	\item Fin del caso de uso.
\end{enumerate}

- - - - \textit{Fin de trayectoria.} \\

%.....................................................
\textbf{\large{Trayectoria alternativa B:}}\\
\textbf{Condición: } El \textit{Entrenador} desea eliminar una rutina que se encuentre asignada a un \textit{Practicante}.

\begin{enumerate}
	\item \includegraphics[width=15pt]{./Figuras/iconosCU/herramienta.png} Muestra el mensaje de error \nameref{msj:MSG09}, indicando que la rutina se encuentra asignada a un \textit{Practicante} y la operación no puede continuar.
	\item Fin del caso de uso.
\end{enumerate}
- - - - \textit{Fin de trayectoria.} \\

%\newpage
\clearpage