\subsubsection{CUE04 Visualizar Practicantes}
\label{cu:CUE04}

\textbf{\textcolor[rgb]{0, 0, 0.545098}{Resumen}} \\

Este caso de uso brinda la posibilidad de visualizar la lista de Practicantes registrados en la herramienta.\\

\textbf{\textcolor[rgb]{0, 0, 0.545098}{Descripción}} \\
%\textbf{Descripción} \\

\begin{table}[H]
\centering
\begin{tabular}{| c | p{12 cm} |}
\hline
\rowcolor[rgb]{0.529412, 0.807843, 0.980392} {\textbf{Caso de uso:}} & \hspace{7em}{\textbf{CUE04 Visualizar Practicantes}}\\
\hline
\textbf{Actor:} &  \nameref{act:Entrenador} \\
\hline
\textbf{Objetivo:} & Visualizar la lista de los Practicantes previamente registrados.\\
\hline
\textbf{Entradas:} & Para llevar a cabo la visualización de la lista de Practicantes, el Entrenador debe hacer uso de la pantalla \nameref{pant:IUE04}.\\
\hline
\textbf{Salidas:} & \vspace{-2mm}	%Para quitar el espacio en blanco de la fila
					\begin{compactitem}
						\setlength\itemsep{-0.25em}
						\item Ninguna.
					\end{compactitem}\\

\hline
\textbf{Precondiciones:} & \vspace{-2mm}	%Para quitar el espacio en blanco de la fila
							\begin{compactitem}
								\setlength\itemsep{-0.25em}
								\item Se requiere que existan elementos en el catálogo: \textit{Practicantes}.
							\end{compactitem}\\
\hline
\textbf{Postcondiciones:} & \vspace{-2mm}	%Para quitar el espacio en blanco de la fila
							\begin{compactitem}
								%\setlength\itemsep{-0.25em}
								\item Ninguna
							\end{compactitem}\\
\hline
\textbf{Reglas de negocio:} & \vspace{-2mm}	%Para quitar el espacio en blanco de la fila
							\begin{compactitem}
								%\setlength\itemsep{-0.25em}
								\item Ninguna.
							\end{compactitem}\\							
\hline
\textbf{Errores:} & \vspace{-2mm}	%Para quitar el espacio en blanco de la fila
					\begin{compactitem}
						\setlength\itemsep{-0.25em}
						\item  Se muestra en el menú principal \nameref{menu:ME02}, el mensaje de error \nameref{msj:MSG24}, cuando cuando debido a un error de conexión no se muestren los elementos de forma correcta.
					\end{compactitem}\\
\hline
\textbf{Tipo:} & Primario.\\
\hline	
\end{tabular}
\caption{Resumen de atributos de CUE04 Visualizar Practicantes}
\label{tab:CUE04}
\end{table} 

%--------------------------------------------------------------------------------------------------------
\textbf{\textcolor[rgb]{0, 0, 0.545098}{Trayectorias del caso de uso}} \\
%\textbf{Trayectorias del caso de uso} \\

\textbf{\large{Trayectoria principal}}

\begin{enumerate}
	\item \includegraphics[width=15pt, height=10pt]{./Figuras/iconosCU/usuario.png} Solicita visualizar la lista de Practicantes seleccionando la opción \textit{Lista de Practicantes} del menú \nameref{menu:ME02}. 
	\item \includegraphics[width=15pt]{./Figuras/iconosCU/herramienta.png} Busca la información del catálogo lista de Practicantes. \textbf{[Trayectoria A]}
	\item \includegraphics[width=15pt]{./Figuras/iconosCU/herramienta.png} Muestra la pantalla \nameref{pant:IUE04}. \textbf{[Trayectoria B]}
\end{enumerate}
	
- - - - \textit{Fin del caso de uso.} \\

%.....................................................
\textbf{\large{Trayectoria alternativa A:}}\\
\textbf{Condición: } Debido a un error de conexión, no se muestran los elementos de forma correcta.

\begin{enumerate}
	\item \includegraphics[width=15pt]{./Figuras/iconosCU/herramienta.png} Muestra en el menú \nameref{menu:ME02}, el mensaje de error \nameref{msj:MSG24}, indicando que la operación no puede continuar de forma correcta.
	\item Fin del caso de uso.
\end{enumerate}

- - - - \textit{Fin de trayectoria.} \\

%.....................................................
\textbf{\large{Trayectoria alternativa B:}}\\
\textbf{Condición: } El \textit{Entrenador} desea regresar a la pantalla anterior.

\begin{enumerate}
	\item \includegraphics[width=15pt, height=10pt]{./Figuras/iconosCU/usuario.png} Solicita regresar a la pantalla anterior presionando el botón \textit{Regresar}.
	\item \includegraphics[width=15pt]{./Figuras/iconosCU/herramienta.png} Muestra el menú \nameref{menu:ME02}.
	\item Fin del caso de uso.
\end{enumerate}

- - - - \textit{Fin de trayectoria.} \\

\textbf{\large{Extiende:}}

\begin{enumerate}
	\item \includegraphics[width=15pt]{./Figuras/iconosCU/herramienta.png} \nameref{cu:CUE04.1}
	\item \includegraphics[width=15pt]{./Figuras/iconosCU/herramienta.png} \nameref{cu:CUE04.2}
	\item \includegraphics[width=15pt]{./Figuras/iconosCU/herramienta.png} \nameref{cu:CUE04.3}
	\item \includegraphics[width=15pt]{./Figuras/iconosCU/herramienta.png} \nameref{cu:CUE04.4}
\end{enumerate}

%\newpage
\clearpage