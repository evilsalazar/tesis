\subsubsection{CUE03 Visualizar rutinas}
\label{cu:CUE03}

\textbf{\textcolor[rgb]{0, 0, 0.545098}{Resumen}} \\

Este caso de uso brinda al Entrenador la posibilidad de visualizar la lista de ejercicios de calentamiento disponibles, la lista de movimientos de técnica capturados y la lista de las rutinas de entrenamiento registradas.\\

\textbf{\textcolor[rgb]{0, 0, 0.545098}{Descripción}}

\begin{table}[H]
\centering
\begin{tabular}{| l | p{12 cm} |}
\hline
\rowcolor[rgb]{0.529412, 0.807843, 0.980392} {\textbf{Caso de uso:}} & \hspace{7em}{\textbf{CUE03 Visualizar rutinas}}\\
\hline
\textbf{Actor:} &  \nameref{act:Entrenador}. \\
\hline
\textbf{Objetivo:} & Permitir al Entrenador visualizar las listas de movimientos y rutinas previamente registrados.\\
\hline
\textbf{Entradas:} & Para llevar a cabo los elementos, el Entrenador debe hacer uso de la pantalla \nameref{pant:IUE03}. \\
\hline
\textbf{Salidas:} & \vspace{-2mm}	%Para quitar el espacio en blanco de la fila
					\begin{compactitem}
						\setlength\itemsep{-0.25em}
						\item Ninguna.
					\end{compactitem}\\
\hline
\textbf{Precondiciones:} & 	\vspace{-2mm}	%Para quitar el espacio en blanco de la fila
							\begin{compactitem}
								\setlength\itemsep{-0.25em}
								\item Se requiere que existan elementos en al menos uno de los catálogos: ejercicios de calentamiento, movimientos de técnica y rutinas de entrenamiento.
							\end{compactitem}\\
\hline
\textbf{Postcondiciones:} & \vspace{-2mm}	%Para quitar el espacio en blanco de la fila
							\begin{compactitem}
								%\setlength\itemsep{-0.25em}
								\item Ninguna.
							\end{compactitem}\\
							
\hline
\textbf{Reglas de negocio:} & \vspace{-2mm}	%Para quitar el espacio en blanco de la fila
							\begin{compactitem}
								%\setlength\itemsep{-0.25em}
								\item Ninguna.
							\end{compactitem}\\							
\hline
\textbf{Errores:} &	\vspace{-2mm}	%Para quitar el espacio en blanco de la fila
					\begin{compactitem}
						\setlength\itemsep{-0.25em}
						\item  Se muestra en el menú principal \nameref{menu:ME01}, el mensaje de error \nameref{msj:MSG24}, cuando debido a un error de conexión no se muestren los elementos de forma correcta.
					\end{compactitem}\\
\hline
\textbf{Tipo:} & Primario.\\
\hline	
\end{tabular}
\caption{Resumen de atributos de CUE03 Visualizar rutinas}
\label{tab:CUE03}
\end{table} 

%--------------------------------------------------------------------------------------------------------
\textbf{\textcolor[rgb]{0, 0, 0.545098}{Trayectorias del caso de uso}} \\
%\textbf{Trayectorias del caso de uso} \\

\textbf{\large{Trayectoria principal}}

\begin{enumerate}
	\item \includegraphics[width=15pt, height=10pt]{./Figuras/iconosCU/usuario.png} Solicita visualizar los ejercicios de calentamiento, movimientos de técnica y rutinas de entrenamiento registrados en la herramienta seleccionando la opción \textit{Rutinas} del menú \nameref{menu:ME01}.
	\item \includegraphics[width=15pt]{./Figuras/iconosCU/herramienta.png} Busca la información en los catálogos de \textit{ejercicios de calentamiento, movimientos de técnica y rutinas de entrenamiento}. \textbf{[Trayectoria A]}
	\item \includegraphics[width=15pt]{./Figuras/iconosCU/herramienta.png} Muestra las listas rutinas, ejercicios de calentamiento y movimientos de técnica en la pantalla \nameref{pant:IUE03}. \textbf{[Trayectoria B]}
\end{enumerate}
	
- - - - \textit{Fin del caso de uso.} \\

%.....................................................
\textbf{\large{Trayectoria alternativa A:}}\\
\textbf{Condición: } Debido a un error de conexión, no se muestran los elementos de forma correcta.

\begin{enumerate}
	\item \includegraphics[width=15pt]{./Figuras/iconosCU/herramienta.png} Muestra en el menú principal \nameref{menu:ME01}, el mensaje de error \nameref{msj:MSG24}, indicando que la operación no puede continuar de forma correcta.
	\item Fin del caso de uso.
\end{enumerate}

- - - - \textit{Fin de trayectoria.} \\

%.....................................................
\textbf{\large{Trayectoria alternativa B:}}\\
\textbf{Condición: } El \textit{Entrenador} desea regresar a la pantalla anterior.

\begin{enumerate}
	\item \includegraphics[width=15pt, height=10pt]{./Figuras/iconosCU/usuario.png} Solicita regresar a la pantalla anterior seleccionando la opción \textit{Regresar}.
	\item \includegraphics[width=15pt]{./Figuras/iconosCU/herramienta.png} Muestra el menú principal \nameref{menu:ME01}.
	\item Fin del caso de uso.
\end{enumerate}

- - - - \textit{Fin de trayectoria.} \\

\textbf{\large{Extiende:}}

\begin{enumerate}
	\item \includegraphics[width=15pt]{./Figuras/iconosCU/herramienta.png} \nameref{cu:CUE03.1}
	\item \includegraphics[width=15pt]{./Figuras/iconosCU/herramienta.png} \nameref{cu:CUE03.2}
	\item \includegraphics[width=15pt]{./Figuras/iconosCU/herramienta.png} \nameref{cu:CUE03.3}
\end{enumerate}
%\newpage
\clearpage