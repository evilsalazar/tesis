\subsubsection{CUE04.4.1 Realizar comentarios}
\label{cu:CUE04.4.1}

\textbf{\textcolor[rgb]{0, 0, 0.545098}{Resumen}} \\

Este caso de uso brinda al Entrenador la posibilidad de ingresar un comentario escrito donde puede realizar una observación acerca del desempeño obtenido por el Practicante.\\

\textbf{\textcolor[rgb]{0, 0, 0.545098}{Descripción}}

\begin{table}[H]
\centering
\begin{tabular}{| l | p{12 cm} |}
\hline
\rowcolor[rgb]{0.529412, 0.807843, 0.980392} {\textbf{Caso de uso:}} & \hspace{7em}{\textbf{CUE04.4.1 Realizar comentarios}}\\
\hline
\textbf{Actor:} &  \nameref{act:Entrenador}. \\
\hline
\textbf{Objetivo:} & Permitir al Entrenador realizar un comentario sobre el desempeño del Practicante.\\
\hline
\textbf{Entradas:} & Para llevar a cabo el registro de un comentario el Entrenador debe hacer uso de la pantalla \nameref{pant:IUE04.4.1} donde debe capturar la siguiente información:
		\begin{compactitem} 
			\setlength\itemsep{-0.25em}
			\item Comentario. Se escribe desde el teclado.
		\end{compactitem} \\
\hline
\textbf{Salidas:} & \vspace{-2mm}	%Para quitar el espacio en blanco de la fila
					\begin{compactitem}
						\setlength\itemsep{-0.25em}
						\item Se muestra en la pantalla \nameref{pant:IUE04.4.1} el mensaje de notificación \nameref{msj:MSG01}, cuando se registre un comentario de manera exitosa.
					\end{compactitem}\\
\hline
\textbf{Precondiciones:} & 	\vspace{-2mm}	%Para quitar el espacio en blanco de la fila
							\begin{compactitem}
								\setlength\itemsep{-0.25em}
								\item Ninguna.
							\end{compactitem}\\
\hline
\textbf{Postcondiciones:} & \vspace{-2mm}	%Para quitar el espacio en blanco de la fila
							\begin{compactitem}
								\setlength\itemsep{-0.25em}
								\item Se registra o actualiza en la herramienta un comentario en la rutina seleccionada.
							\end{compactitem}\\
\hline
\textbf{Reglas de negocio:} & \vspace{-2mm}	%Para quitar el espacio en blanco de la fila
							\begin{compactitem}
								\setlength\itemsep{-0.25em}
								\item \nameref{rn:RNR25}.
								\item \nameref{rn:RNR26}.
								\item \nameref{rn:RNR27}.
								
							\end{compactitem}\\							
\hline
\textbf{Errores:} &	\vspace{-2mm}	%Para quitar el espacio en blanco de la fila
					\begin{compactitem}
						\setlength\itemsep{-0.25em}
						\item Se muestra en la pantalla \nameref{pant:IUE04.4.1} el mensaje de error \nameref{msj:MSG13} cuando el formato del comentario ingresado por el Entrenador sea incorrecto.
					\end{compactitem}\\
\hline
\textbf{Tipo:} & Secundario, viene del \nameref{cu:CUE04.4}.\\
\hline	
\end{tabular}
\caption{Resumen de atributos de CUE04.4.1 Realizar comentario}
\label{tab:CUE04.4.1}
\end{table} 

%--------------------------------------------------------------------------------------------------------
\textbf{\textcolor[rgb]{0, 0, 0.545098}{Trayectorias del caso de uso}} \\
%\textbf{Trayectorias del caso de uso} \\

\textbf{\large{Trayectoria principal}}

\begin{enumerate}
	\item \includegraphics[width=15pt, height=10pt]{./Figuras/iconosCU/usuario.png} Solicita ingresar un comentario seleccionando la opción \textit{Realizar comentario} de la pantalla \nameref{pant:IUE04.4}.
	\item \includegraphics[width=15pt]{./Figuras/iconosCU/herramienta.png} Verifica que se cumpla lo especificado en la regla de negocio \nameref{rn:RNR26}.
	\item \includegraphics[width=15pt]{./Figuras/iconosCU/herramienta.png} Verifica que se cumpla lo especificado en la regla de negocio \nameref{rn:RNR27}.
	\item \includegraphics[width=15pt]{./Figuras/iconosCU/herramienta.png} Muestra la pantalla \nameref{pant:IUE04.4.1}.
	\item \includegraphics[width=15pt, height=10pt]{./Figuras/iconosCU/usuario.png} Ingresa el comentario.
	\item \includegraphics[width=15pt]{./Figuras/iconosCU/herramienta.png} Verifica que el comentario cumpla con el formato especificado en la regla de negocio \nameref{rn:RNR25}. \textbf{[Trayectoria A]}
	\item \includegraphics[width=15pt, height=10pt]{./Figuras/iconosCU/usuario.png} Solicita guardar el comentario seleccionando la opción \textit{Guardar} de la pantalla \nameref{pant:IUE04.4.1}. \textbf{[Trayectoria B]}
	\item \includegraphics[width=15pt]{./Figuras/iconosCU/herramienta.png} Registra el nuevo comentario de la rutina realizada en la herramienta. 
	\item \includegraphics[width=15pt]{./Figuras/iconosCU/herramienta.png} Muestra el mensaje de notificación \nameref{msj:MSG01}.
	\item \includegraphics[width=15pt]{./Figuras/iconosCU/herramienta.png} Regresa a la pantalla \nameref{pant:IUE04.4}.
\end{enumerate}
	
- - - - \textit{Fin del caso de uso.} \\


%.....................................................
\textbf{\large{Trayectoria alternativa A:}}\\
\textbf{Condición: } El comentario no cumple con el formato especificado.

\begin{enumerate}
	\item \includegraphics[width=15pt]{./Figuras/iconosCU/herramienta.png} Muestra en la pantalla \nameref{pant:IUE04.4.1}, el mensaje de error \nameref{msj:MSG13}
	\item Continua en el paso 5 de la trayectoria principal.
\end{enumerate}

- - - - \textit{Fin de trayectoria.} \\

%.....................................................
\textbf{\large{Trayectoria alternativa B:}}\\
\textbf{Condición: } El \textit{Entrenador} desea cancelar la operación.

\begin{enumerate}
	\item \includegraphics[width=15pt, height=10pt]{./Figuras/iconosCU/usuario.png} Solicita cancelar la operación seleccionando la opción \textit{Cancelar}.
	\item \includegraphics[width=15pt]{./Figuras/iconosCU/herramienta.png} Cancela la operación y regresa a la pantalla \nameref{pant:IUE04.4}.
	\item Fin del caso de uso.
\end{enumerate}

\clearpage