\subsubsection{CUP02 Visualizar datos personales}
\label{cu:CUP02}

\textbf{\textcolor[rgb]{0, 0, 0.545098}{Resumen}} \\

Este caso de uso brinda al Practicante la posibilidad de visualizar sus datos personales.\\

\textbf{\textcolor[rgb]{0, 0, 0.545098}{Descripción}}

\begin{table}[H]
\centering
\begin{tabular}{| l | p{12 cm} |}
\hline
\rowcolor[rgb]{0.529412, 0.807843, 0.980392} {\textbf{Caso de uso:}} & \hspace{7em}{\textbf{CUP02 Visualizar datos personales}}\\
\hline
\textbf{Actor:} &  \nameref{act:Practicante}. \\
\hline
\textbf{Objetivo:} & Permitir al Practicante visualizar sus datos personales. \\
\hline
\textbf{Entradas:} & Ninguna. \\
\hline
\textbf{Salidas:} &	Ninguna. \\
\hline
\textbf{Precondiciones:} & 	\vspace{-2mm}	%Para quitar el espacio en blanco de la fila
							\begin{compactitem}
							\setlength\itemsep{-0.25em}
								\item Se requiere que el Practicante haya iniciado sesión previamente.
							\end{compactitem} \\
\hline
\textbf{Postcondiciones:} & \vspace{-2mm}	%Para quitar el espacio en blanco de la fila
							\begin{compactitem}
								%\setlength\itemsep{-0.25em}
								\item Ninguna.
							\end{compactitem}\\
							
\hline
\textbf{Reglas de negocio:} & \vspace{-2mm}	%Para quitar el espacio en blanco de la fila
							\begin{compactitem}
								%\setlength\itemsep{-0.25em}
								\item Ninguna.
							\end{compactitem}\\							
\hline
\textbf{Errores:} &	\vspace{-2mm}	%Para quitar el espacio en blanco de la fila
							\begin{compactitem}
								%\setlength\itemsep{-0.25em}
								\item Ninguno.
							\end{compactitem}\\
\hline
\textbf{Tipo:} & Primario.\\
\hline	
\end{tabular}
\caption{Resumen de atributos de CUP02 Visualizar datos personales}
\label{tab:CUP02}
\end{table} 

%--------------------------------------------------------------------------------------------------------
\textbf{\textcolor[rgb]{0, 0, 0.545098}{Trayectorias del caso de uso}} \\

\textbf{\large{Trayectoria principal}}

\begin{enumerate}
	\item \includegraphics[width=15pt, height=10pt]{./Figuras/iconosCU/usuario.png} Solicita visualizar sus datos personales registrados en la herramienta seleccionando la opción \textit{Datos personales} del menú \nameref{menu:MP01}. \textbf{[Trayectoria A]}
	\item \includegraphics[width=15pt]{./Figuras/iconosCU/herramienta.png} Busca la información en el catálogo \textit{Practicantes}.
	\item \includegraphics[width=15pt]{./Figuras/iconosCU/herramienta.png} Muestra  los datos personales del \textit{Practicante} en la pantalla \nameref{pant:IUP02}. \textbf{[Trayectoria B]}
\end{enumerate}
	
- - - - \textit{Fin del caso de uso.} \\

%.....................................................
\textbf{\large{Trayectoria alternativa A:}} \\
\textbf{Condición: } El Practicante desea cerrar sesión.

\begin{enumerate}
	\item \includegraphics[width=15pt, height=10pt]{./Figuras/iconosCU/usuario.png} Solicita cerrar sesión seleccionando la opción \textit{Cerrar sesión} del menú principal \nameref{menu:MP01}.
	\item \includegraphics[width=15pt]{./Figuras/iconosCU/herramienta.png} Cierra la sesión del \textit{Practicante}.
	\item Fin del caso de uso.
\end{enumerate}

- - - - \textit{Fin de trayectoria.} \\

%.....................................................
\textbf{\large{Trayectoria alternativa B:}} \\
\textbf{Condición: } El Practicante desea regresar al menú principal.

\begin{enumerate}
	\item \includegraphics[width=15pt, height=10pt]{./Figuras/iconosCU/usuario.png} Solicita regresar seleccionando la opción \textit{Regresar} de la pantalla \nameref{pant:IUP02}.
	\item \includegraphics[width=15pt]{./Figuras/iconosCU/herramienta.png} Muestra el menú principal \nameref{menu:MP01}.
	\item \includegraphics[width=15pt]{./Figuras/iconosCU/herramienta.png} Continúa en el paso 1 de la trayectoria principal.
\end{enumerate}

- - - - \textit{Fin de trayectoria.} \\

%.....................................................
\textbf{\large{Extiende:}}\\

\begin{enumerate}
	\item \includegraphics[width=15pt]{./Figuras/iconosCU/herramienta.png} \nameref{cu:CUP02.1}.
\end{enumerate}
\clearpage