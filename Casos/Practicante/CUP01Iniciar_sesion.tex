\subsubsection{CUP01 Iniciar sesión}
\label{cu:CUP01}

\textbf{\textcolor[rgb]{0, 0, 0.545098}{Resumen}} \\

Este caso de uso permite al Practicante iniciar sesión en la herramienta. \\

\textbf{\textcolor[rgb]{0, 0, 0.545098}{Descripción}} \\

\begin{table}[H]
\centering
\begin{tabular}{| c | p{12 cm} |}
\hline
\rowcolor[rgb]{0.529412, 0.807843, 0.980392} {\textbf{Caso de uso:}} & \hspace{7em}{\textbf{CUP01 Iniciar sesión}}\\
\hline
\textbf{Actor:} & \nameref{act:Practicante} \\
\hline
\textbf{Objetivo:} & Permitir al Practicante ingresar a la herramienta.\\
\hline
\textbf{Entradas:} & Para llevar a cabo el inicio de sesión, el Practicante debe hacer uso de la pantalla \nameref{pant:IUP01} y debe ingresar los siguientes datos: 
	\begin{compactitem} 
			\setlength\itemsep{-0.25em}
			\item Nombre de usuario. Se escribe desde el teclado.
			\item Contraseña. Se escribe desde el teclado.
	\end{compactitem} \\
\hline
\textbf{Salidas:} & \vspace{-2mm}	%Para quitar el espacio en blanco de la fila
					\begin{compactitem}
						\setlength\itemsep{-0.25em}
						\item Ninguna
					\end{compactitem}\\
\hline
\textbf{Precondiciones:} & \vspace{-2mm}	%Para quitar el espacio en blanco de la fila
							\begin{compactitem}
								\setlength\itemsep{-0.25em}
								\item Se requiere que el Practicante se encuentre registrado en el catálogo: \textit{Practicantes}
							\end{compactitem}\\
\hline
\textbf{Postcondiciones:} & \vspace{-2mm}	%Para quitar el espacio en blanco de la fila
							\begin{compactitem}
								%\setlength\itemsep{-0.25em}
								\item Ninguna.
							\end{compactitem}\\
\hline
\textbf{Reglas de negocio:} & \vspace{-2mm}	%Para quitar el espacio en blanco de la fila
							\begin{compactitem}
								%\setlength\itemsep{-0.25em}
								\item \nameref{rn:RNR32}.
							\end{compactitem}\\	
\hline
\textbf{Errores:} & \vspace{-2mm}	%Para quitar el espacio en blanco de la fila
					\begin{compactitem}
						\setlength\itemsep{-0.25em}
						\item Se muestra en pantalla el mensaje de error \nameref{msj:MSG10}, cuando el Practicante haya ingresado alguno de los datos de manera incorrecta.
					\end{compactitem}\\
\hline
\textbf{Tipo:} & Primario\\
\hline	
\end{tabular}
\caption{Resumen de atributos de CUP01 Iniciar sesión}
\label{tab:CUP01}
\end{table} 

%--------------------------------------------------------------------------------------------------------
\textbf{\textcolor[rgb]{0, 0, 0.545098}{Trayectorias del caso de uso}} \\
%\textbf{Trayectorias del caso de uso} \\

\textbf{\large{Trayectoria principal}} \\

\begin{enumerate}
	\item \includegraphics[width=15pt]{./Figuras/iconosCU/herramienta.png} Solicita al \textit{Practicante} ingresar los datos de inicio de sesión mediante la pantalla \nameref{pant:IUP01}.
	\item \includegraphics[width=15pt, height=10pt]{./Figuras/iconosCU/usuario.png} Ingresa en los campos su nombre de usuario y su contraseña.
	\item \includegraphics[width=15pt, height=10pt]{./Figuras/iconosCU/usuario.png} Solicita ingresar a la herramienta seleccionando la opción \textit{Iniciar sesión}.
	\item \includegraphics[width=15pt]{./Figuras/iconosCU/herramienta.png} Verifica que la información ingresada cumpla con lo especificado en la regla de negocio \nameref{rn:RNR32}.
	\item \includegraphics[width=15pt]{./Figuras/iconosCU/herramienta.png} Verifica que la información ingresada coincida con la información de sesión de los Practicantes. \textbf{[Trayectoria A]}
	\item \includegraphics[width=15pt]{./Figuras/iconosCU/herramienta.png} Muestra el menú \nameref{menu:MP01}.
\end{enumerate}
	
- - - - \textit{Fin del caso de uso.} \\


%.....................................................
\textbf{\large{Trayectoria alternativa A:}} \\
\textbf{Condición: } El \textit{Practicante} ingresó valores incorrectos.

\begin{enumerate}
	\item \includegraphics[width=15pt]{./Figuras/iconosCU/herramienta.png} Muestra en la pantalla \nameref{pant:IUP01} el mensaje de error \nameref{msj:MSG10}.
	\item Fin del caso de uso.
\end{enumerate}

- - - - \textit{Fin de trayectoria.} \\

%.....................................................
\textbf{\large{Extiende:}}\\

\begin{enumerate}
	\item \includegraphics[width=15pt]{./Figuras/iconosCU/herramienta.png} \nameref{cu:CUP01.1}.
\end{enumerate}
\clearpage