\subsubsection{CUP02.1 Modificar contraseña}
\label{cu:CUP02.1}

\textbf{\textcolor[rgb]{0, 0, 0.545098}{Resumen}} \\

Este caso de uso brinda al Practicante la posibilidad de registrar una nueva contraseña. \\

\textbf{\textcolor[rgb]{0, 0, 0.545098}{Descripción}} \\
%\textbf{Descripción} \\

\begin{table}[H]
\centering
\begin{tabular}{| c | p{12 cm} |}
\hline
\rowcolor[rgb]{0.529412, 0.807843, 0.980392}  {\textbf{Caso de uso:}} & \hspace{7em}{\textbf{CUP02.1: Modificar contraseña}}\\
\hline
\textbf{Actor:} & \nameref{act:Practicante} \\
\hline
\textbf{Objetivo:} & Permitir al usuario registrar una nueva contraseña de ingreso a la herramienta.\\
\hline
\textbf{Entradas:} & Para llevar a cabo la modificación de la contraseña, el Practicante debe hacer uso de la pantalla \nameref{pant:IUP02.1} y debe capturar la siguiente información:
	\begin{compactitem} 
			\setlength\itemsep{-0.25em}
			\item Nueva contraseña. Se escribe desde el teclado.
			\item Confirmación de contraseña. Se escribe desde el teclado.
	\end{compactitem} \\
\hline
\textbf{Salidas:} & \vspace{-2mm}	%Para quitar el espacio en blanco de la fila
					\begin{compactitem}
						\setlength\itemsep{-0.25em}
						\item Se muestra en la pantalla \nameref{pant:IUP02.1} el mensaje de notificación \nameref{msj:MSG01} indicando que la herramienta ha actualizado la contraseña de manera exitosa.
					\end{compactitem}\\
\hline
\textbf{Precondiciones:} & \vspace{-2mm}	%Para quitar el espacio en blanco de la fila
							\begin{compactitem}
								\setlength\itemsep{-0.25em}
								\item Ninguna.
							\end{compactitem}\\
\hline
\textbf{Postcondiciones:} & \vspace{-2mm}	%Para quitar el espacio en blanco de la fila
							\begin{compactitem}
								%\setlength\itemsep{-0.25em}
								\item Se actualizará el catálogo: lista de Practicantes con la nueva información.
							\end{compactitem}\\
\hline
\textbf{Reglas de negocio:} & \vspace{-2mm}	%Para quitar el espacio en blanco de la fila
							\begin{compactitem}
								%\setlength\itemsep{-0.25em}
								\item \nameref{rn:RNR01}.
								\item \nameref{rn:RNR09}.
								\item \nameref{rn:RNR31}.
							\end{compactitem}\\	
\hline
\textbf{Errores:} & \vspace{-2mm}	%Para quitar el espacio en blanco de la fila
					\begin{compactitem}
						\setlength\itemsep{-0.25em}
						\item Se muestra en la pantalla \nameref{pant:IUP02.1} el mensaje de error \nameref{msj:MSG12} cuando el Practicante no haya ingresado campos obligatorios.
						% \item Se muestra en pantalla el mensaje de error \nameref{msj:MSG13}, cuando el Practicante haya ingresado datos con un formato incorrecto en algún campo.
					\end{compactitem}\\							
\hline
\end{tabular}
\end{table} 

\begin{table}[H]
\centering
\begin{tabular}{| c | p{12 cm} |}
\hline
\textbf{Errores:} & \vspace{-2mm}	%Para quitar el espacio en blanco de la fila
					\begin{compactitem}
						\setlength\itemsep{-0.25em}
						% \item Se muestra en la pantalla \nameref{pant:IUP02.1} el mensaje de error \nameref{msj:MSG12} cuando el Practicante no haya ingresado campos obligatorios.
						\item Se muestra en pantalla el mensaje de error \nameref{msj:MSG13}, cuando el Practicante haya ingresado datos con un formato incorrecto en algún campo.
						\item Se mostrará en pantalla el mensaje de error \nameref{msj:MSG14} cuando el Practicante haya ingresado contraseñas que no coinciden en ambos campos.
					\end{compactitem}\\
\hline
\textbf{Tipo:} & Secundario, viene del caso de uso \nameref{cu:CUP02}.\\
\hline	
\end{tabular}
\caption{Resumen de atributos de CUP02.1 Modificar contraseña}
\label{tab:CUP02.1}
\end{table} 

%--------------------------------------------------------------------------------------------------------
\textbf{\textcolor[rgb]{0, 0, 0.545098}{Trayectorias del caso de uso}} \\

\textbf{\large{Trayectoria principal}} \\

\begin{enumerate}
	\item \includegraphics[width=15pt, height=10pt]{./Figuras/iconosCU/usuario.png} Solicita modificar su contraseña seleccionando la opción \textit{Modificar contraseña} mediante el uso de la pantalla \nameref{pant:IUP02}. 
	\item \includegraphics[width=15pt]{./Figuras/iconosCU/herramienta.png} Solicita al \textit{Practicante} ingresar los datos requeridos para modificar su contraseña mediante la pantalla \nameref{pant:IUP02.1}.
	\item \includegraphics[width=15pt, height=10pt]{./Figuras/iconosCU/usuario.png} Ingresa los datos requeridos.
	\item \includegraphics[width=15pt, height=10pt]{./Figuras/iconosCU/usuario.png} Solicita guardar la nueva contraseña seleccionando la opción \textit{Guardar}. \textbf{[Trayectoria A]}
	\item \includegraphics[width=15pt]{./Figuras/iconosCU/herramienta.png} Verifica que se hayan ingresado los valores en los campos requeridos, según lo especificado en la regla de negocio \nameref{rn:RNR31}. \textbf{[Trayectoria B]}
	\item \includegraphics[width=15pt]{./Figuras/iconosCU/herramienta.png} Verifica que la contraseña cumpla con el formato correcto, según lo especificado en la regla de negocio \nameref{rn:RNR01} y \nameref{rn:RNR09}. \textbf{[Trayectoria C]}
	\item \includegraphics[width=15pt]{./Figuras/iconosCU/herramienta.png} Verifica que la contraseña ingresada en los dos campos coincida. \textbf{[Trayectoria D]}
	\item \includegraphics[width=15pt]{./Figuras/iconosCU/herramienta.png} Registra la nueva contraseña.
	\item \includegraphics[width=15pt]{./Figuras/iconosCU/herramienta.png} Muestra en pantalla el mensaje de notificación \nameref{msj:MSG01}.
	\item \includegraphics[width=15pt]{./Figuras/iconosCU/herramienta.png} Muestra la pantalla \nameref{pant:IUP02}.
\end{enumerate}
	
- - - - \textit{Fin del caso de uso.} \\

%.....................................................
\textbf{\large{Trayectoria alternativa A:}} \\
\textbf{Condición: } El \textit{Practicante} desea cancelar la operación.

\begin{enumerate}
	\item \includegraphics[width=15pt, height=10pt]{./Figuras/iconosCU/usuario.png} Solicita cancelar la operación seleccionando la opción \textit{Cancelar}.
	\item \includegraphics[width=15pt]{./Figuras/iconosCU/herramienta.png} Cancela la operación y regresa a la pantalla \nameref{pant:IUP02}.
	\item Fin del caso de uso.
\end{enumerate}
	

- - - - \textit{Fin de trayectoria.} \\

%.....................................................
\textbf{\large{Trayectoria alternativa B:}} \\
\textbf{Condición: } El \textit{Practicante} no ingresó los valores correspondientes en los campos marcados como obligatorios.

\begin{enumerate}
	\item \includegraphics[width=15pt]{./Figuras/iconosCU/herramienta.png} Muestra en la pantalla \nameref{pant:IUP02.1} el mensaje de error \nameref{msj:MSG12}.
	\item \includegraphics[width=15pt]{./Figuras/iconosCU/herramienta.png} Continúa en el paso 2 de la trayectoria principal.
\end{enumerate}

- - - - \textit{Fin de trayectoria.} \\

%.....................................................
\textbf{\large{Trayectoria alternativa C:}} \\
\textbf{Condición: } El \textit{Practicante} ingresó valores con un formato incorrecto.

\begin{enumerate}
	\item \includegraphics[width=15pt]{./Figuras/iconosCU/herramienta.png} Muestra en pantalla el mensaje de error \nameref{msj:MSG13}.
	\item \includegraphics[width=15pt]{./Figuras/iconosCU/herramienta.png} Continúa en el paso 2 de la trayectoria principal.
\end{enumerate}

- - - - \textit{Fin de trayectoria.} \\

%.....................................................
\textbf{\large{Trayectoria alternativa D:}} \\
\textbf{Condición: } El \textit{Practicante} ingresó dos contraseñas que no coinciden.

\begin{enumerate}
	\item \includegraphics[width=15pt]{./Figuras/iconosCU/herramienta.png} Muestra en pantalla el mensaje de error \nameref{msj:MSG14}.
	\item \includegraphics[width=15pt]{./Figuras/iconosCU/herramienta.png} Continúa en el paso 2 de la trayectoria principal.
\end{enumerate}

- - - - \textit{Fin de trayectoria.} \\
\clearpage