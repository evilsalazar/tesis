\subsubsection{CUP03 Visualizar lista de rutinas}
\label{cu:CUP03}

\textbf{\textcolor[rgb]{0, 0, 0.545098}{Resumen}} \\

Este caso de uso brinda al Practicante la posibilidad de visualizar las rutinas que el Entrenador le haya asignado.\\

\textbf{\textcolor[rgb]{0, 0, 0.545098}{Descripción}}

\begin{table}[H]
\centering
\begin{tabular}{| l | p{12 cm} |}
\hline
\rowcolor[rgb]{0.529412, 0.807843, 0.980392} {\textbf{Caso de uso:}} & \hspace{7em}{\textbf{CUP03 Visualizar lista de rutinas}}\\
\hline
\textbf{Actor:} &  \nameref{act:Practicante}. \\
\hline
\textbf{Objetivo:} & Permitir al Practicante visualizar las rutinas asignadas por el Entrenador, en la semana correspondiente.\\
\hline
\textbf{Entradas:} & Para llevar a cabo la visualización de indicadores de desempeño, el Practicante debe hacer uso de la pantalla \nameref{pant:IUP03}. \\
\hline
\textbf{Salidas:} &	\vspace{-2mm}	%Para quitar el espacio en blanco de la fila
					\begin{compactitem}
					\setlength\itemsep{-0.25em}
						\item Ninguna.
					\end{compactitem}\\
\hline
\textbf{Precondiciones:} & 	\vspace{-2mm}	%Para quitar el espacio en blanco de la fila
							\begin{compactitem}
							\setlength\itemsep{-0.25em}
								\item Se requiere que el Practicante tenga al menos una rutina asignada por parte del Entrenador en el catálogo: \textit{rutinas asignadas}.
							\end{compactitem}\\
\hline
\textbf{Postcondiciones:} & \vspace{-2mm}	%Para quitar el espacio en blanco de la fila
							\begin{compactitem}
								%\setlength\itemsep{-0.25em}
								\item Ninguna.
							\end{compactitem}\\
							
\hline
\textbf{Reglas de negocio:} & \vspace{-2mm}	%Para quitar el espacio en blanco de la fila
							\begin{compactitem}
								%\setlength\itemsep{-0.25em}
								\item Ninguna.
							\end{compactitem}\\							
\hline
\textbf{Errores:} &	\vspace{-2mm}	%Para quitar el espacio en blanco de la fila
							\begin{compactitem}
								%\setlength\itemsep{-0.25em}
								\item Se muestra en el menú principal \nameref{menu:MP01}, el mensaje de error \nameref{msj:MSG24}, cuando cuando debido a un error de conexión no se muestren los elementos de forma correcta.
							\end{compactitem}\\
\hline
\textbf{Tipo:} & Primario.\\
\hline	
\end{tabular}
\caption{Resumen de atributos de CUP03 Visualizar lista de rutinas}
\label{tab:CUP03}
\end{table} 

%--------------------------------------------------------------------------------------------------------
\textbf{\textcolor[rgb]{0, 0, 0.545098}{Trayectorias del caso de uso}} \\

\textbf{\large{Trayectoria principal}}

\begin{enumerate}
	\item \includegraphics[width=15pt, height=10pt]{./Figuras/iconosCU/usuario.png} Solicita visualizar la lista de rutinas disponibles seleccionando la opción \textit{Lista de rutinas} del menú \nameref{menu:MP01}.
	\item \includegraphics[width=15pt]{./Figuras/iconosCU/herramienta.png} Busca la información de los catálogos de \textit{rutinas de entrenamiento}. \textbf{[Trayectoria A]}
	\item \includegraphics[width=15pt]{./Figuras/iconosCU/herramienta.png} Muestra las primeras 3 rutinas asignadas al Practicante en la pantalla \nameref{pant:IUP03}. \textbf{[Trayectoria B]}
\end{enumerate}
	
- - - - \textit{Fin del caso de uso.} \\

%.....................................................
\textbf{\large{Trayectoria alternativa A:}}\\
\textbf{Condición: } Debido a un error de conexión, no se muestran los elementos de forma correcta.

\begin{enumerate}
	\item \includegraphics[width=15pt]{./Figuras/iconosCU/herramienta.png} Muestra en el menú principal \nameref{menu:MP01}, el mensaje de error \nameref{msj:MSG24}, indicando que la operación no puede continuar de forma correcta.
	\item Fin del caso de uso.
\end{enumerate}

- - - - \textit{Fin de trayectoria.} \\

%.....................................................
\textbf{\large{Trayectoria alternativa B:}}\\
\textbf{Condición: } El \textit{Entrenador} desea regresar a la pantalla anterior.

\begin{enumerate}
	\item \includegraphics[width=15pt, height=10pt]{./Figuras/iconosCU/usuario.png} Solicita regresar a la pantalla anterior seleccionando la opción \textit{Regresar}.
	\item \includegraphics[width=15pt]{./Figuras/iconosCU/herramienta.png} Muestra el menú principal \nameref{menu:MP01}.
	\item Fin del caso de uso.
\end{enumerate}

- - - - \textit{Fin de trayectoria.} \\

%.....................................................
\textbf{\large{Extiende:}}\\

\begin{enumerate}
	\item \includegraphics[width=15pt]{./Figuras/iconosCU/herramienta.png} \nameref{cu:CUP03.1}.
	\item \includegraphics[width=15pt]{./Figuras/iconosCU/herramienta.png} \nameref{cu:CUP03.2}.
\end{enumerate}
\clearpage